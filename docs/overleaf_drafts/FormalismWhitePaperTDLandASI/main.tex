\documentclass[preprint, onecolumn, amsmath, amssymb, aps, floatfix]{revtex4-2}

% --- Packages ---
\usepackage[utf8]{inputenc}
% Toggle: uncomment [draft] for fast compile (figures as bounding boxes)
%\usepackage[draft]{graphicx}
\usepackage{graphicx}
\usepackage{booktabs}
\usepackage{bm}
\usepackage{hyperref}
\usepackage{amsthm}
\usepackage{amsmath}
\usepackage{amssymb}
% tikz removed -- diagrams pre-compiled to standalone PDFs (tikz_triangle, tikz_torus)
\usepackage{subcaption}
% longtable omitted -- revtex4-2 provides its own longtable support
\usepackage{xcolor}
\usepackage[capitalize,noabbrev]{cleveref}

% --- Figure path ---
\graphicspath{{figures/}}

% --- Custom environments ---
\newtheorem{keyresult}{Key Result}
\newtheorem{convention}{Convention}
\newtheorem{correspondence}{Correspondence}
\newtheorem{theorem}{Theorem}
\newtheorem{corollary}[theorem]{Corollary}

% --- cleveref formats for custom environments ---
\crefname{keyresult}{Key Result}{Key Results}
\Crefname{keyresult}{Key Result}{Key Results}

% --- Custom operators ---
\DeclareMathOperator{\rank}{rank}
\DeclareMathOperator{\im}{im}
\DeclareMathOperator{\diag}{diag}
\DeclareMathOperator{\tr}{tr}
\DeclareMathOperator{\sgn}{sgn}
\DeclareMathOperator{\perm}{perm}
\DeclareMathOperator{\Skew}{skew}

% --- Shortcuts ---
\newcommand{\R}{\mathbb{R}}
\newcommand{\Z}{\mathbb{Z}}
\newcommand{\calP}{\mathcal{P}}
\newcommand{\calI}{\mathcal{I}}
\newcommand{\calQ}{\mathcal{Q}}
\newcommand{\calH}{\mathcal{H}}

\begin{document}

% ===================================================================
% FRONT MATTER
% ===================================================================
\title{Topological Deep Learning and Artificial Spin Ice:\\ A Mathematical Correspondence and Neural Sampling Proposal}
\author{Carl Merrigan}
\author{Cristiano Nisoli}
\author{Claude Opus-4.6}
\date{\today}

\begin{abstract}
This document has two parts, both preliminary.
%
The first part works out the mathematical correspondence between the
cell-complex formalism used in Topological Deep Learning (TDL)---boundary
operators, Hodge Laplacians, cochains---and Nisoli's charge framework for
artificial spin ice (ASI).  The operators turn out to be identical, not
merely analogous: the same vertex--edge incidence matrix~$B_1$, graph
Laplacian~$L_0$, and lower Hodge Laplacian~$L_1^{\text{down}}$ that
appear in TDL message passing are exactly the objects in the ASI charge
and Hamiltonian theory.  We derive six ``key results'' connecting the two
frameworks and then extend Nisoli's charge Hamiltonian to a \emph{Hodge
Hamiltonian} that penalizes both topological charge ($B_1\bm\sigma$,
monopoles) and topological current ($B_2^T\bm\sigma$, vortices),
unifying both sectors through the full Hodge 1-Laplacian~$L_1$.
A spanning-tree argument yields the per-sector bound
$|\mathcal{I}_{\mathbf{Q}}| \leq 2^{\beta_1}$
on ice states within each charge sector.  We prove a GF(2) bijection
theorem showing that the $\bm\alpha$ loop-flip parameterization is an exact
bijection between ice states and a subset of $\{0,1\}^{\beta_1}$, and
invoke Veblen's theorem to establish that the ice manifold within each
charge sector is connected under directed cycle moves.
Two tables collect the core quantitative results: a counting table
exhibiting the hierarchy
$|\varepsilon_{\mathbf{Q}}| \leq |\mathcal{I}_{\mathbf{Q}}| \leq 2^{\beta_1}$---where
$|\varepsilon_{\mathbf{Q}}|$ is the subset reachable by basis-restricted
sequential sampling, falling below the topological bound by up to five
orders of magnitude across a zoo of eight lattice types---and a topological
invariants catalog listing every simplex count and Betti number under
both face-filling strategies and both boundary conditions.  Worked
examples with explicit Laplacian matrices appear in the appendices.
%
The second part sketches an idea for future application: using recent
work on variational autoregressive networks (VAN), message-passing VAN
(MPVAN), and the EIGN edge-level GNN architecture to build neural samplers
for spin ice configurations.  We outline two complementary modes---loop-basis
sampling (Mode~A, guaranteed ice-rule compliance) and direct-edge sampling
(Mode~B, finite temperature)---and write down the layer update equations
and training objectives.  These architectures are at a preliminary concept
stage; there are many implementation and training details still to be
worked out, and we expect the design to evolve considerably.
We share this document to invite critical review of the ideas before
investing further in implementation.
\end{abstract}

\maketitle

% ===================================================================
% FIGURES PAGE: LATTICE GALLERY
% ===================================================================
\clearpage

\begin{figure*}[p]
\centering
\begin{subfigure}[t]{0.32\textwidth}
  \includegraphics[width=\linewidth]{square_4x4_open_graph}
  \caption{Graph structure (open)}
\end{subfigure}
\hfill
\begin{subfigure}[t]{0.32\textwidth}
  \includegraphics[width=\linewidth]{square_4x4_open_orientation}
  \caption{$B_1$ orientation (open)}
\end{subfigure}
\hfill
\begin{subfigure}[t]{0.32\textwidth}
  \includegraphics[width=\linewidth]{square_4x4_open_ice_state}
  \caption{Ice state (open)}
\end{subfigure}

\medskip

\begin{subfigure}[t]{0.32\textwidth}
  \includegraphics[width=\linewidth]{square_4x4_periodic_graph}
  \caption{Graph structure (periodic)}
\end{subfigure}
\hfill
\begin{subfigure}[t]{0.32\textwidth}
  \includegraphics[width=\linewidth]{square_4x4_periodic_orientation}
  \caption{$B_1$ orientation (periodic)}
\end{subfigure}
\hfill
\begin{subfigure}[t]{0.32\textwidth}
  \includegraphics[width=\linewidth]{square_4x4_periodic_ice_state}
  \caption{Ice state (periodic)}
\end{subfigure}
\caption{Square lattice ($4\!\times\!4$). Top row: open boundary conditions
  ($n_0 = 16$, $n_1 = 24$, $\beta_1 = 9$). Bottom row: periodic boundary
  conditions ($n_0 = 16$, $n_1 = 32$, $\beta_1 = 17$). Left: graph
  structure with vertices colored by coordination number. Center: reference
  edge orientations defining $B_1$. Right: a representative ice-rule
  configuration.}
\label{fig:square}
\end{figure*}

\begin{figure*}[p]
\centering
\begin{subfigure}[t]{0.32\textwidth}
  \includegraphics[width=\linewidth]{kagome_2x2_open_graph}
  \caption{Graph structure (open)}
\end{subfigure}
\hfill
\begin{subfigure}[t]{0.32\textwidth}
  \includegraphics[width=\linewidth]{kagome_2x2_open_orientation}
  \caption{$B_1$ orientation (open)}
\end{subfigure}
\hfill
\begin{subfigure}[t]{0.32\textwidth}
  \includegraphics[width=\linewidth]{kagome_2x2_open_ice_state}
  \caption{Ice state (open)}
\end{subfigure}

\medskip

\begin{subfigure}[t]{0.32\textwidth}
  \includegraphics[width=\linewidth]{kagome_2x2_periodic_graph}
  \caption{Graph structure (periodic)}
\end{subfigure}
\hfill
\begin{subfigure}[t]{0.32\textwidth}
  \includegraphics[width=\linewidth]{kagome_2x2_periodic_orientation}
  \caption{$B_1$ orientation (periodic)}
\end{subfigure}
\hfill
\begin{subfigure}[t]{0.32\textwidth}
  \includegraphics[width=\linewidth]{kagome_2x2_periodic_ice_state}
  \caption{Ice state (periodic)}
\end{subfigure}
\caption{Kagome lattice ($2\!\times\!2$). All bulk vertices have coordination
  $z = 3$ (odd), so the ice rule allows $|Q_v| = 1$. Top: open BC
  ($n_1 = 7$, $\beta_1 = 0$, 20 charge sectors). Bottom: periodic BC
  ($n_1 = 12$, $\beta_1 = 5$, 70 charge sectors).}
\label{fig:kagome}
\end{figure*}

\begin{figure*}[p]
\centering
\begin{subfigure}[t]{0.32\textwidth}
  \includegraphics[width=\linewidth]{santa_fe_2x2_open_graph}
  \caption{Graph structure}
\end{subfigure}
\hfill
\begin{subfigure}[t]{0.32\textwidth}
  \includegraphics[width=\linewidth]{santa_fe_2x2_open_orientation}
  \caption{$B_1$ orientation}
\end{subfigure}
\hfill
\begin{subfigure}[t]{0.32\textwidth}
  \includegraphics[width=\linewidth]{santa_fe_2x2_open_ice_state}
  \caption{Ice state}
\end{subfigure}
\caption{Santa Fe lattice ($2\!\times\!2$, open BC). Mixed coordination
  $z \in \{2,3,4\}$. $n_1 = 30$, $\beta_1 = 7$, 68 feasible charge sectors.
  Polymer-like strings of unhappy vertices are characteristic of this geometry.}
\label{fig:santa_fe}
\end{figure*}

\begin{figure*}[p]
\centering
\begin{subfigure}[t]{0.32\textwidth}
  \includegraphics[width=\linewidth]{shakti_2x2_open_graph}
  \caption{Graph structure}
\end{subfigure}
\hfill
\begin{subfigure}[t]{0.32\textwidth}
  \includegraphics[width=\linewidth]{shakti_2x2_open_orientation}
  \caption{$B_1$ orientation}
\end{subfigure}
\hfill
\begin{subfigure}[t]{0.32\textwidth}
  \includegraphics[width=\linewidth]{shakti_2x2_open_ice_state}
  \caption{Ice state}
\end{subfigure}
\caption{Shakti lattice ($2\!\times\!2$, open BC). Mixed coordination
  $z \in \{2,3,4\}$ with maximal vertex frustration. $n_1 = 81$,
  $\beta_1 = 18$, $\geq 1{,}881$ charge sectors. Extensive degeneracy and
  topological order~\cite{morrison2013}.}
\label{fig:shakti}
\end{figure*}

\begin{figure*}[p]
\centering
\begin{subfigure}[t]{0.32\textwidth}
  \includegraphics[width=\linewidth, height=0.18\textheight, keepaspectratio]{tetris_2x2_open_graph}
  \caption{Graph structure ($2\!\times\!2$)}
\end{subfigure}
\hfill
\begin{subfigure}[t]{0.32\textwidth}
  \includegraphics[width=\linewidth, height=0.18\textheight, keepaspectratio]{tetris_2x2_open_orientation}
  \caption{$B_1$ orientation ($2\!\times\!2$)}
\end{subfigure}
\hfill
\begin{subfigure}[t]{0.32\textwidth}
  \includegraphics[width=\linewidth, height=0.18\textheight, keepaspectratio]{tetris_2x2_open_ice_state}
  \caption{Ice state ($2\!\times\!2$)}
\end{subfigure}

\medskip

\begin{subfigure}[t]{0.32\textwidth}
  \includegraphics[width=\linewidth, height=0.18\textheight, keepaspectratio]{tetris_3x3_open_graph}
  \caption{Graph structure ($3\!\times\!3$)}
\end{subfigure}
\hfill
\begin{subfigure}[t]{0.32\textwidth}
  \includegraphics[width=\linewidth, height=0.18\textheight, keepaspectratio]{tetris_3x3_open_orientation}
  \caption{$B_1$ orientation ($3\!\times\!3$)}
\end{subfigure}
\hfill
\begin{subfigure}[t]{0.32\textwidth}
  \includegraphics[width=\linewidth, height=0.18\textheight, keepaspectratio]{tetris_3x3_open_ice_state}
  \caption{Ice state ($3\!\times\!3$)}
\end{subfigure}
\caption{Tetris lattice (open BC). Mixed coordination $z \in \{2,3,4\}$ with
  maximal vertex frustration. Top: $2\!\times\!2$ ($n_1 = 38$, $\beta_1 = 7$).
  Bottom: $3\!\times\!3$ ($n_1 = 93$, $\beta_1 = 22$). Tetris exhibits the
  most extreme directed-cycle reduction: $|\varepsilon_{\mathbf{q}}|/2^{\beta_1}
  = 6 \times 10^{-4}\%$ at $3\!\times\!3$.}
\label{fig:tetris}
\end{figure*}

\clearpage

% ===================================================================
% SECTION 1: INTRODUCTION
% ===================================================================
\section{Introduction}
\label{sec:intro}

This document works through the mathematical correspondence between two
bodies of theory that, at first glance, belong to entirely different fields:

\begin{itemize}
  \item \textbf{Topological Deep Learning (TDL)}---the framework of cell
    complexes, boundary operators, Hodge Laplacians, and cochains that
    generalizes graph neural networks to higher-order
    structures~\cite{hajij2023topological,papillon2024architectures}.
  \item \textbf{Artificial Spin Ice (ASI) graph theory}---the charge
    framework~\cite{nisoli2020} for frustrated lattice systems, which describes
    spin configurations, topological charges, ice rules, and entropic
    interactions using the graph Laplacian and its pseudoinverse.
\end{itemize}

The connection turns out to be structural, not analogical: the same
operator---the graph Laplacian~$L_0$ and its edge-level counterpart
$L_1^{\text{down}} = B_1^T B_1$---governs both the entropic interactions
between topological charges in spin ice \emph{and} the message-passing
dynamics in graph neural networks.  The first part of this document
(Secs.~\ref{sec:notation}--\ref{sec:hamiltonian}) spells out the dictionary
in detail, deriving six ``key results'' that connect objects across the two
frameworks---$S_{vv'}$, topological charge, Coulomb classes, and the ice
manifold from ASI physics; boundary operators, the Hodge decomposition,
and the Hodge Laplacian from TDL.
The key operators are: the vertex--edge incidence matrix~$B_1$, which
encodes which vertices bound each edge; the edge--face incidence
matrix~$B_2$, which encodes which edges bound each face; the graph
(vertex) Laplacian $L_0 = B_1 B_1^T$, governing vertex-level
interactions; the lower Hodge Laplacian
$L_1^{\text{down}} = B_1^T B_1$, governing edge-level interactions
through shared vertices; the upper Hodge Laplacian
$L_1^{\text{up}} = B_2 B_2^T$, governing edge-level interactions
through shared faces; the full Hodge 1-Laplacian
$L_1 = L_1^{\text{down}} + L_1^{\text{up}}$; and the face Laplacian
$L_2 = B_2^T B_2$.  These are defined formally in
\cref{sec:boundary_ops,sec:topology_zoo}.
A particular focus is the counting table (\cref{sec:counting_table}), which
tracks how much the directed-cycle constraint reduces the reachable ice-state
count below the topological bound $2^{\beta_1}$.

Two tables collect the core quantitative results.
\Cref{tab:counting} (the ``counting table'') tracks how the directed-cycle
constraint prunes the reachable ice-state count, revealing a five-order-of-magnitude
variation across the lattice zoo.  \Cref{tab:topology_zoo} catalogs every
simplex count and Betti number under both face-filling strategies and both
boundary conditions, verifying the Euler relation concretely for all lattices.

The second part (\cref{sec:outlook}) sketches an idea for where this
correspondence might lead: neural samplers for spin ice configurations,
built by combining three recent developments from the machine learning
literature.  Variational autoregressive networks
(VANs)~\cite{wu2019variational} showed that autoregressive neural networks
can learn to sample from Boltzmann distributions of classical spin systems,
replacing MCMC with single-pass generation.  The message-passing VAN
(MPVAN)~\cite{ma2024mpvan} demonstrated that incorporating the Hamiltonian's
coupling structure directly into message-passing weights improves sampling
quality on frustrated Ising models.  And the EIGN
architecture~\cite{fuchsgruber2025} introduced edge-level message passing
that respects orientation equivariance, using operator channels built from
$B_1$ and $|B_1|$.  \Cref{sec:outlook} sketches how these three ideas might
combine into two complementary sampling architectures for ASI---one operating
in the loop basis (Mode~A), the other sampling edges directly (Mode~B).
These are preliminary designs; there are likely many implementation and
training details that will need to be worked out, and the architectures may
change substantially as we learn more.

The ice manifold (the degenerate ground-state manifold of a frustrated spin
ice) is structurally governed by $\ker(B_1)$---either as the ice manifold
itself (for even-degree vertices) or as the space of fluctuations within it
(for odd-degree vertices).  The dimension of this space, its combinatorial
density (Pauling entropy), and its spectral structure all vary across lattice
types in ways that ASI physics has characterized in detail.  A particular
focus of the formalism sections is the relationship between $\beta_1$ and
the degeneracy of the ice manifold.  A key finding is the global bound
on the total ice manifold,
$|\calI| \leq |\calQ_{\min}|\cdot 2^{\beta_1}$
(\cref{eq:global_bound}), derived from a spanning-tree argument that
yields the per-sector bound $|\calI_{\mathbf{Q}}| \leq 2^{\beta_1}$
and then sums over all $|\calQ_{\min}|$ feasible charge sectors.
The actual reachable count within each sector is controlled by a
\textbf{directed-cycle constraint} that can reduce it by orders of magnitude
(\cref{sec:counting_table}).  This distinction matters for neural sampling,
where training difficulty depends on the reachable state count, not on
$\beta_1$ alone.

The remainder of this document is organized as follows.
\Cref{sec:notation} establishes notation and the cell complex setup.
\Cref{sec:S_matrix,sec:charge} derive the correspondence for the
$S_{vv'}$ matrix and the topological charge.
\Cref{sec:coulomb} develops the Coulomb class structure, ice manifold
parameterization, and the counting table.
\Cref{sec:hodge} presents the Hodge decomposition, and \cref{sec:hamiltonian}
connects Nisoli's charge Hamiltonian to the lower
Hodge Laplacian $L_1^{\text{down}}$, then extends it to a
\emph{Hodge Hamiltonian} that penalizes both topological charge
($B_1\bm\sigma$, monopoles) and topological current
($B_2^T\bm\sigma$, vortices), unifying both sectors through the full
$L_1$.
\Cref{sec:outlook} sketches the two neural-sampling architectures, and
\cref{sec:summary} summarizes.

% ===================================================================
% SECTION 2: NOTATION & SETUP
% ===================================================================
\section{Notation and Setup}
\label{sec:notation}

\subsection{The graph}
\label{sec:graph}

We work with a \textbf{connected} simple undirected graph $G = (V, E)$ with
$n_0 = |V|$ vertices and $n_1 = |E|$ edges. Connectivity means $\beta_0 = 1$
(one connected component); the graph Laplacian $L_0$ has exactly one zero
eigenvalue with eigenvector $\mathbf{1}$, and $\rank(B_1) = n_0 - 1$. All
ASI lattices we consider are connected, planar (or toroidal under periodic
boundary conditions), with vertex coordination numbers $z_v \in \{2,3,4\}$.

To define incidence matrices and boundary operators, we equip each edge with
a \textbf{reference orientation}: for edge~$e$ connecting vertices~$u$
and~$v$, we designate one vertex as the tail (source) and the other as the
head (target). The physics does not depend on this choice---changing the
orientation of an edge negates the corresponding column of $B_1$ and the
corresponding component of any 1-cochain, leaving all observables invariant.

\begin{convention}[Reference Orientation]
For lattices on $\Z^2$, we orient horizontal edges left$\to$right and
vertical edges bottom$\to$top. For edges not aligned with axes, we orient
from the vertex with smaller index to the vertex with larger index.
\end{convention}
\noindent
Concrete reference orientations for every lattice in the zoo are shown in
\cref{fig:square,fig:kagome,fig:santa_fe,fig:shakti,fig:tetris} (center
panels).

\subsection{Cell complex}
\label{sec:simplicial}

A cell complex $X$ built on $G$ extends the graph to include
higher-dimensional cells:
\begin{itemize}
  \item \textbf{0-cells (vertices):} the $n_0$ vertices of $G$
  \item \textbf{1-cells (edges):} the $n_1$ oriented edges of $G$
  \item \textbf{2-cells (faces):} $n_2$ oriented filled polygonal faces
\end{itemize}
When all faces are triangles, $X$ is a simplicial complex;
for lattices with square or larger faces (e.g., kagome hexagons), $X$ is a cell complex.  The
boundary operators, Hodge Laplacians, and Hodge decomposition are defined
identically in both cases.

The vertices and edges are given by the graph, but the \textbf{faces are a
modeling choice}. Morrison, Nelson \& Nisoli~\cite{morrison2013} define a
\emph{minimal loop} as a closed chain of edges that does not contain vertices
in its interior---equivalently, the faces of the planar embedding. These
minimal loops are the fundamental objects in ASI frustration theory and the
natural candidates for 2-cells. Given the set of all minimal loops, we
consider two bracketing strategies:

\begin{table}[h]
\centering
\caption{Face-filling strategies and their effect on $\beta_1$.}
\label{tab:face_strategies}
\begin{tabular}{llll}
\toprule
\textbf{Strategy} & $n_2$ & $L_1^{\text{up}}$ & \textbf{Effect on $\beta_1$} \\
\midrule
Fill all faces & Maximal & Nonzero & Minimal $\beta_1$ \\
Fill no faces  & $0$     & $0$     & Maximal $\beta_1 = n_1 - n_0 + 1$ \\
\bottomrule
\end{tabular}
\end{table}

\textbf{Boundary conditions} also have a dramatic effect on $\beta_1$.
Consider the $3\times 3$ square lattice ($n_0=9$, $n_1^{\text{per}}=18$,
$n_1^{\text{open}}=12$):

\begin{table}[h]
\centering
\caption{Boundary condition effects on the $3\times 3$ square lattice.}
\label{tab:bc_effects}
\begin{tabular}{lccc}
\toprule
\textbf{Property} & \textbf{Periodic (torus)} & \textbf{Open (planar)} \\
\midrule
$n_0$ (vertices)                   & 9   & 9 \\
$n_1$ (edges)                      & 18  & 12 \\
$n_2$ (faces, all filled)          & 9   & 4 \\
$\chi = n_0 - n_1 + n_2$          & 0   & 1 \\
$\beta_0$                          & 1   & 1 \\
$\beta_2$                          & 1   & 0 \\
$\beta_1$ (all faces filled)       & \textbf{2}  & \textbf{0} \\
$\beta_1$ (no faces filled)        & \textbf{10} & \textbf{4} \\
\bottomrule
\end{tabular}
\end{table}

On the periodic torus, $\beta_1 = 2$ with all faces filled corresponds to the
two independent winding modes (horizontal and vertical) of $H_1(\mathbb{T}^2)
\cong \Z^2$. All three horizontal row-loops are \emph{homologous}---they
differ by the boundary of the faces between them---and similarly for the three
vertical column-loops. In the open case, every cycle on the planar patch bounds
a region of faces, so $\beta_1 = 0$ with all faces filled. This makes boundary
conditions a critical parameter: \textbf{periodic BCs guarantee a nonzero
$\beta_1$ floor} (at least~2 for a torus).

\subsection{Notation correspondence table}
\label{sec:notation_table}

\Cref{tab:notation} provides a dictionary between TDL and Nisoli notation.

\begin{table*}[t]
\centering
\caption{Notation correspondence between TDL and Nisoli's ASI charge framework.}
\label{tab:notation}
\small
\begin{tabular}{llll}
\toprule
\textbf{Object} & \textbf{TDL Notation} & \textbf{Nisoli Notation} & \textbf{Notes} \\
\midrule
Undirected graph & $G = (V,E)$ & $G$ with $N_v$ vertices, $N_l$ edges & $n_0 = N_v$, $n_1 = N_l$ \\
Adjacency matrix & $A$ or $A_{0,\text{down}}$ & $A_{vv'}$ & Symmetric, binary \\
Degree / coordination & $\deg(v)$, degree matrix $D$ & $z_v$ & $D_{vv} = z_v$ \\
Signed incidence & $B_1$ ($n_0 \times n_1$) & (implicit in $S_{vv'}$) & $[B_1]_{v,e} = \pm 1$ \\
Edge-face incidence & $B_2$ ($n_1 \times n_2$) & (not used by Nisoli) & Hodge decomposition \\
Edge signal / 1-cochain & $\mathbf{f} \in C^1(X;\R)$ & $\sigma_l$ (spin on edge) & $\sigma_e \in \{+1,-1\}$ \\
Vertex signal / 0-cochain & $\mathbf{h} \in C^0(X;\R)$ & $\varphi_v$ (entropic field) & \\
Graph Laplacian & $L_0 = B_1 B_1^T$ & $\hat{L}_{vv'} = z_v\delta_{vv'} - A_{vv'}$ & $L_0 = D - A$ \\
Lower Hodge Laplacian & $L_1^{\text{down}} = B_1^T B_1$ & (in energy: $\bm\sigma^T L_1^{\text{down}}\bm\sigma$) & Acts on $\R^{n_1}$ \\
Upper Hodge Laplacian & $L_1^{\text{up}} = B_2 B_2^T$ & (not used by Nisoli) & Requires 2-cells \\
Full Hodge 1-Laplacian & $L_1 = L_1^{\text{down}} + L_1^{\text{up}}$ & --- & Central for oversmoothing \\
Antisymmetric spin matrix & --- & $S_{vv'} = A^{\text{dir}}_{vv'} - A^{\text{dir}}_{v'v}$ & See \cref{sec:S_matrix} \\
Topological charge & $(B_1\bm\sigma)_v$ & $Q_v[S] = \sum_{v'} S_{v'v}$ & See \cref{sec:charge} \\
Ice manifold & $\ker(B_1) \cap \{\pm 1\}^{n_1}$ & $\calI = \{\sigma: Q_v = 0\;\forall v\}$ & See \cref{sec:ice_manifold} \\
Harmonic subspace & $\ker(L_1)$, dim $= \beta_1$ & --- & See \cref{sec:hodge} \\
\bottomrule
\end{tabular}
\end{table*}

\subsection{Boundary operators}
\label{sec:boundary_ops}

The \textbf{vertex-edge incidence matrix} $B_1$ ($n_0 \times n_1$) is defined
by: for edge $e$ oriented from vertex $u$ (tail) to $v$ (head),
\begin{equation}
  [B_1]_{w,e} = \begin{cases}
    +1 & \text{if } w = v \text{ (head of } e\text{)} \\
    -1 & \text{if } w = u \text{ (tail of } e\text{)} \\
    0  & \text{otherwise.}
  \end{cases}
  \label{eq:B1def}
\end{equation}

Just as each edge requires a reference orientation to define $B_1$, each
face requires a \textbf{reference orientation}---a choice of traversal
direction around its boundary cycle.

\begin{convention}[Face Orientation]
We orient every face counterclockwise (CCW).
\end{convention}
\noindent
The entry $[B_2]_{e,f}$ then records whether
edge~$e$'s reference orientation agrees or disagrees with the CCW
traversal of face~$f$:

The \textbf{edge-face incidence matrix} $B_2$ ($n_1 \times n_2$) is defined
by: for face $f$ with CCW-oriented boundary cycle,
\begin{equation}
  [B_2]_{e,f} = \begin{cases}
    +1 & \text{if } e \in \partial f \text{ and its reference orientation}\\
       & \text{agrees with the CCW traversal of } f \\
    -1 & \text{if } e \in \partial f \text{ and its reference orientation}\\
       & \text{opposes the CCW traversal of } f \\
    0  & \text{if } e \notin \partial f.
  \end{cases}
  \label{eq:B2def}
\end{equation}
As with edge orientations, the physics does not depend on the face
orientation convention: reversing a face negates the corresponding column
of $B_2$, leaving all Laplacians and observables invariant.

The fundamental chain complex property is
\begin{equation}
  B_1 B_2 = 0,
  \label{eq:chain}
\end{equation}
which states that ``the boundary of a boundary is zero.'' Equivalently, the
divergence of a curl is zero, or $\im(B_2) \subseteq \ker(B_1)$.

% ===================================================================
% SECTION 3: THE S MATRIX
% ===================================================================
\section{The \texorpdfstring{$S_{vv'}$}{S} Matrix as Incidence Times Spin Configuration}
\label{sec:S_matrix}

\subsection{Nisoli's definition}

In Nisoli~\cite{nisoli2020}, a spin configuration on a graph assigns to each
edge a direction. This is encoded in the directed adjacency matrix:
\begin{equation}
  A^{\text{dir}}_{vv'} = \begin{cases}
    1 & \text{if the spin on edge } \{v,v'\} \text{ points } v \to v' \\
    0 & \text{otherwise.}
  \end{cases}
  \label{eq:Adir}
\end{equation}
For each undirected edge $\{v,v'\}$, exactly one of $A^{\text{dir}}_{vv'}$ and
$A^{\text{dir}}_{v'v}$ equals~1. Nisoli defines the antisymmetric matrix:
\begin{equation}
  S_{vv'} = A^{\text{dir}}_{vv'} - A^{\text{dir}}_{v'v},
  \label{eq:Sdef}
\end{equation}
satisfying $S_{vv'} = -S_{v'v} \in \{-1, 0, +1\}$, with $S_{vv'} = 0$ for
non-adjacent vertices.

\subsection{TDL representation of spin configurations}

In TDL, each edge $e$ has a fixed reference orientation encoded in $B_1$.
A spin configuration is a vector $\bm\sigma \in \{+1,-1\}^{n_1}$ where
\begin{equation}
  \sigma_e = \begin{cases}
    +1 & \text{spin aligns with reference orientation of } e \\
    -1 & \text{spin opposes reference orientation of } e.
  \end{cases}
  \label{eq:sigmadef}
\end{equation}

\subsection{Derivation of the correspondence}

Consider edge $e$ from $u$ (tail) to $v$ (head), so $[B_1]_{v,e} = +1$ and
$[B_1]_{u,e} = -1$. A case analysis on $\sigma_e = \pm 1$ yields, for any
pair of vertices $w, w'$ sharing edge $e$:
\begin{equation}
  S_{ww'} = -[B_1]_{w,e}\,\sigma_e.
  \label{eq:Selement}
\end{equation}
Antisymmetry follows immediately: since $[B_1]_{w,e} + [B_1]_{w',e} = 0$
for any edge $e$ connecting $w$ and $w'$,
\[
  S_{ww'} + S_{w'w} = -\sigma_e\bigl([B_1]_{w,e} + [B_1]_{w',e}\bigr) = 0.
\]

\subsection{Matrix form}

A matrix expression for $S$ requires breaking the symmetry of
$B_1 D_\sigma B_1^T$ (which is symmetric). Using the \textbf{unsigned
incidence matrix} $|B_1|$ (entry-wise absolute value) on one side, define
$M = |B_1|\,\diag(\bm\sigma)\,B_1^T$. Its transpose is
$M^T = B_1\,\diag(\bm\sigma)\,|B_1|^T \neq M$. The skew-symmetrization
gives $S$ exactly:

\begin{keyresult}[$S_{vv'}$ from TDL Operators]
\label{kr:S_matrix}
Nisoli's antisymmetric spin matrix is determined by $B_1$ and
$\bm\sigma$ via:

\emph{Element-wise:}
\begin{equation}
  S_{ww'} = -[B_1]_{w,e}\,\sigma_e,
\end{equation}
where $e$ is the unique edge connecting $w$ and $w'$.

\emph{Matrix form} (skew-symmetrization):
\begin{equation}
  S = \frac{1}{2}\Bigl(|B_1|\,\diag(\bm\sigma)\,B_1^T
      - B_1\,\diag(\bm\sigma)\,|B_1|^T\Bigr).
  \label{eq:Smatrix}
\end{equation}
\end{keyresult}

\noindent
\textbf{Proof of~\eqref{eq:Smatrix}.}
For adjacent vertices $u$ (tail) and $v$ (head) sharing edge $e$:
\[
  [|B_1|\,D_\sigma\,B_1^T]_{uv}
  = |{[B_1]_{u,e}}|\,\sigma_e\,[B_1]_{v,e}
  = (1)\,\sigma_e\,(+1) = +\sigma_e,
\]
\[
  [B_1\,D_\sigma\,|B_1|^T]_{uv}
  = [B_1]_{u,e}\,\sigma_e\,|{[B_1]_{v,e}}|
  = (-1)\,\sigma_e\,(1) = -\sigma_e.
\]
Therefore $S_{uv} = \tfrac{1}{2}(\sigma_e - (-\sigma_e)) = \sigma_e$,
matching~\eqref{eq:Selement}. The diagonal vanishes automatically:
the two terms contribute identical diagonal entries, which cancel in the
subtraction.~$\square$

% ===================================================================
% SECTION 4: TOPOLOGICAL CHARGE
% ===================================================================
\section{Topological Charge as Divergence of 1-Cochain}
\label{sec:charge}

\subsection{Nisoli's charge definition}

Nisoli~\cite{nisoli2020} defines the topological charge at vertex $v$ as
\begin{equation}
  Q_v[S] = \sum_{v'} S_{v'v}
  = \#(\text{spins into } v) - \#(\text{spins out of } v).
  \label{eq:Qdef_nisoli}
\end{equation}
He also defines a divergence $\mathrm{div}[S]_v = \sum_{v'} S_{vv'} = -Q_v$,
counting outflow rather than inflow.

\subsection{Derivation}

We prove that $(B_1\bm\sigma)_v = Q_v$ by expanding the matrix-vector product:
\begin{equation}
  (B_1\bm\sigma)_v = \sum_{e=1}^{n_1} [B_1]_{v,e}\,\sigma_e
  = \sum_{\substack{e:\\ v = \text{head}(e)}} \sigma_e
  - \sum_{\substack{e:\\ v = \text{tail}(e)}} \sigma_e.
  \label{eq:B1sigma}
\end{equation}
Each edge incident to $v$ contributes $+1$ if its spin points into $v$
and $-1$ if its spin points out of $v$, regardless of the reference
orientation convention.

\begin{keyresult}[Charge as Boundary]
\label{kr:charge}
\begin{equation}
  \boxed{Q_v = (B_1\,\bm\sigma)_v = \sum_e [B_1]_{v,e}\,\sigma_e}
\end{equation}
The topological charge vector is the image of the spin configuration under
the incidence matrix: $\mathbf{Q} = B_1\bm\sigma$. The ice rule
$Q_v = 0\;\forall v$ is equivalent to $\bm\sigma \in \ker(B_1)$.
\end{keyresult}

Total charge is always conserved: $\sum_v Q_v = \mathbf{1}^T B_1\bm\sigma = 0$
since each column of $B_1$ sums to zero.

\subsection{The $M$ matrix encodes both $S$ and $\mathbf{Q}$}

The matrix $M = |B_1|\,D_\sigma\,B_1^T$ from \cref{eq:Smatrix} encodes
both of Nisoli's fundamental objects:
\begin{equation}
  \diag(M) = \mathbf{Q}, \qquad \Skew(M) = S.
\end{equation}
The skew-symmetrization that extracts $S$ discards the charges; the diagonal
extraction that gives $\mathbf{Q}$ discards the spin connectivity. The two
are complementary projections of the same underlying object.

% ===================================================================
% SECTION 5: COULOMB CLASSES AND ICE MANIFOLD
% ===================================================================
\section{Coulomb Classes, the Ice Manifold, and Degeneracy}
\label{sec:coulomb}

\subsection{Phase space and charge equivalence}
\label{sec:phase_space}

The phase space of a spin ice on graph $G$ is
$\calP = \{+1,-1\}^{n_1}$ with $|\calP| = 2^{n_1}$.
Two configurations $\bm\sigma, \bm\sigma' \in \calP$ are
\textbf{charge-equivalent} if
\begin{equation}
  \bm\sigma \sim \bm\sigma' \;\iff\;
  B_1\bm\sigma = B_1\bm\sigma' \;\iff\;
  \bm\sigma - \bm\sigma' \in \ker(B_1).
  \label{eq:charge_equiv}
\end{equation}

\subsection{Coulomb classes as affine cosets}
\label{sec:coulomb_classes}

The equivalence relation~\eqref{eq:charge_equiv} partitions $\calP$ into
\textbf{Coulomb classes} (charge sectors) labeled by the charge vector
$\mathbf{q} = B_1\bm\sigma$:
\begin{equation}
  C_{\mathbf{q}} = \bigl\{\bm\sigma \in \{+1,-1\}^{n_1}
  : B_1\bm\sigma = \mathbf{q}\bigr\}
  = \bigl(\bm\sigma_0 + \ker(B_1)\bigr) \cap \{+1,-1\}^{n_1},
  \label{eq:coulomb_class}
\end{equation}
where $\bm\sigma_0$ is any representative. The phase space partitions as
\begin{equation}
  \calP = \bigsqcup_{\mathbf{q} \in \calQ} C_{\mathbf{q}},
\end{equation}
with charge conservation $\sum_v q_v = 0$ always enforced.

\subsection{The ice manifold}
\label{sec:ice_manifold}

The ice rule at vertex $v$ requires that $|Q_v|$ be minimized. Parity
dictates $|Q_v|_{\min} = z_v \bmod 2$. The ice manifold is
\begin{equation}
  \calI = \bigl\{\bm\sigma \in \{+1,-1\}^{n_1}
  : |Q_v| = z_v \bmod 2 \;\;\forall v\bigr\}.
  \label{eq:ice_manifold}
\end{equation}

\textbf{Even-degree vertices} ($z_v = 2,4,\ldots$): the ice rule requires
$Q_v = 0$ exactly, and
$\calI_{\text{even}} = \{+1,-1\}^{n_1} \cap \ker(B_1)$.

\textbf{Odd-degree vertices} ($z_v = 3,5,\ldots$): the best achievable is
$Q_v = \pm 1$. These residual charges are topologically mandated by parity,
not excitations. Ice configurations in a given charge sector form an affine
coset:
\begin{equation}
  \calI_{\mathbf{q}} = \{\bm\sigma: B_1\bm\sigma = \mathbf{q}\}
  = \bm\sigma_0 + \ker(B_1).
\end{equation}
The full ice manifold is a union over minimal-charge sectors:
$\calI = \bigcup_{\mathbf{q} \in \calQ_{\min}} \calI_{\mathbf{q}}$.

Although ice configurations in the odd-degree case do not lie in $\ker(B_1)$,
the \emph{difference} between any two configurations in the same charge sector
does: $\bm\sigma_1 - \bm\sigma_2 \in \ker(B_1)$. So \textbf{fluctuations
within the ice manifold} are always divergence-free.

For even-degree graphs, $\calI = C_{\mathbf{0}}$ is a single Coulomb class
(Nisoli's ``Coulomb phase''). For odd-degree graphs (e.g., kagome with $z=3$),
the ice manifold is a union of multiple Coulomb classes.

\subsection{The fluctuation space $\ker(B_1)$}
\label{sec:kerB1}

The divergence-free subspace always has dimension
\begin{equation}
  \dim(\ker B_1) = n_1 - \rank(B_1) = n_1 - n_0 + 1
  \label{eq:dim_kerB1}
\end{equation}
for connected graphs ($\beta_0 = 1$). This is a topological invariant
depending only on $n_1$ and $n_0$ (equivalently, the mean coordination
$\bar{z}$).

\subsection{Continuous dimension vs.\ discrete degeneracy}
\label{sec:continuous_vs_discrete}

The \textbf{continuous dimension} $\dim(\ker B_1) = n_1 - n_0 + 1$ is the
dimension of the real vector space of divergence-free edge flows. The
\textbf{discrete degeneracy} $|\calI| = |\ker(B_1) \cap \{\pm 1\}^{n_1}|$ is
a combinatorial quantity that depends on the full graph structure. These are
fundamentally different.

\paragraph{Pauling entropy.}
The entropy per vertex of the ice manifold is
\begin{equation}
  s(G) = \frac{\ln|\calI|}{n_0}.
  \label{eq:pauling_entropy}
\end{equation}
This varies across lattice types even when $\dim(\ker B_1)$ is the same.

\paragraph{The Pauling approximation.}
Start with $|\calP| = 2^{n_1}$ and multiply by the probability that a random
spin assignment satisfies the ice rule at each vertex (treating constraints as
independent):
\begin{equation}
  |\calI_{\text{P}}| = 2^{n_1}\prod_{v \in V} f(z_v),
  \label{eq:pauling_approx}
\end{equation}
where the per-vertex ice-rule fraction is
\begin{equation}
  f(z_v) = \begin{cases}
    \displaystyle\frac{\binom{z_v}{z_v/2}}{2^{z_v}}
      & \text{if } z_v \text{ even}, \\[8pt]
    \displaystyle\frac{2\,\binom{z_v}{\lfloor z_v/2\rfloor}}{2^{z_v}}
      & \text{if } z_v \text{ odd}.
  \end{cases}
\end{equation}
The key values: $f(2) = 1/2$, $f(3) = 3/4$, $f(4) = 3/8$.
For the square lattice (all $z_v = 4$):
$|\calI_{\text{P}}| = 2^{2n_0}(3/8)^{n_0} = (3/2)^{n_0}$,
giving $s_{\text{P}} = \ln(3/2) \approx 0.405$ per vertex, remarkably close
to Lieb's exact result~\cite{lieb1967}
$s_{\text{Lieb}} = \tfrac{3}{2}\ln(4/3) \approx 0.431$.

\paragraph{The permanent formula.}
For even-degree graphs, Caravelli et al.~\cite{caravelli2021}
establish the exact formula
$|\calI| = \perm(A) / \prod_v (d_v/2)!$,
where $A$ is an incidence-derived matrix.
Computing $\perm(A)$ is \#P-complete---the ice degeneracy is provably hard
to determine exactly for general graphs.

\paragraph{Hierarchy of bounds} for $z=4$ periodic square lattices:

\begin{table}[h]
\centering
\caption{Hierarchy of entropy per vertex $s = \ln|\cdot|/n_0$ for ice-state
counts ($z=4$ periodic square, no faces filled so
$\beta_1 = n_0 + 1$).  All entries are thermodynamic limits
($n_0 \to \infty$).}
\label{tab:hierarchy}
\begin{tabular}{lll}
\toprule
\textbf{Estimate} & \textbf{$s$ (per vertex)} & \textbf{Type} \\
\midrule
$|\calP| = 2^{n_1}$  & $2\ln 2 \approx 1.386$ & Trivial upper bound \\
$2^{\beta_1}$ ($n_2=0$) & $\ln 2 \approx 0.693$ & Topological upper bound \\
Lieb exact            & $\tfrac{3}{2}\ln\tfrac{4}{3} \approx 0.431$ & Exact in TDL \\
Pauling $|\calI_{\text{P}}|$ & $\ln\tfrac{3}{2} \approx 0.405$ & Mean-field approximation \\
\bottomrule
\end{tabular}
\end{table}

Each gap is exponential in $n_0$.

\paragraph{Why $2^{\beta_1}$ is a valid upper bound (per charge sector).}
The bound follows from a \textbf{spanning-tree argument}
(illustrated in \cref{fig:spanning_tree} for the all-even case).
Fix a feasible charge vector~$\mathbf{Q}$ (i.e.\ $Q_v = 0$ for
even-degree vertices, $Q_v \in \{\pm 1\}$ for odd-degree vertices,
$\sum_v Q_v = 0$).  Every ice state in sector~$\mathbf{Q}$ satisfies
$B_1\bm\sigma = \mathbf{Q}$.
\begin{enumerate}
  \item Pick a spanning tree $T$ of the graph ($n_0 - 1$ edges).  The
    remaining $\beta_1 = n_1 - n_0 + 1$ edges are \emph{cotree edges}.
  \item Each cotree edge can independently take values $\pm 1$, giving
    $2^{\beta_1}$ choices.
  \item For each choice, the constraint $B_1\bm\sigma = \mathbf{Q}$
    \textbf{uniquely determines} every tree-edge value.  Root~$T$ at an
    arbitrary vertex and process vertices bottom-up (leaves first).  At
    each non-root vertex~$v$, the child tree edges and all cotree edges
    incident to~$v$ have already been fixed; the parent tree edge is the
    sole unknown.  The equation $(B_1\bm\sigma)_v = Q_v$ is then one
    linear equation in one unknown whose coefficient is $\pm 1$, so it has
    a unique real solution.  The root equation is redundant---it is
    automatically satisfied because every column of $B_1$ sums to zero, so
    $\sum_v(B_1\bm\sigma)_v = 0 = \sum_v Q_v$ for any~$\bm\sigma$.  This
    gives $n_0 - 1$ independent equations for the $n_0 - 1$ tree-edge
    unknowns, each solved in turn.
  \item However, the determined tree-edge values need not all land in
    $\{\pm 1\}$---on interior vertices with high connectivity, the
    propagation can produce values $\pm 3$, $\pm 5$, etc.  Only cotree
    assignments whose propagation stays within $\{\pm 1\}$ yield valid ice
    states.
\end{enumerate}

\begin{figure*}[t]
  \centering
  \includegraphics[width=\textwidth]{spanning_tree_argument}
  \caption{Spanning-tree argument on the $3\times 3$ periodic square lattice
    ($n_0=9$, $n_1=18$, $\beta_1=10$).
    \textbf{(a)}~The lattice graph with interior (solid) and wrap-around (dashed) edges.
    \textbf{(b)}~An S-shaped spanning tree (dark blue, 8 edges) and its cotree (orange, 10 edges);
    circled numbers show the tree-path order.
    \textbf{(c)}~Cotree assignment: independently assign $\sigma\in\{\pm1\}$ to each cotree edge
    (blue $=+1$, red $=-1$); tree edges remain unknown.
    \textbf{(d)}~Bottom-up propagation via the ice rule uniquely determines every
    tree edge (circled numbers give the solve order), yielding a valid ice state.
    \textbf{(e)}~Invalid state: a different cotree assignment (all $+1$) produces
    tree-edge values $\pm 3$ and $\pm 5$ (red labels), none in $\{\pm1\}$---this
    assignment does not extend to an ice state.}
  \label{fig:spanning_tree}
\end{figure*}

Therefore $|\calI_{\mathbf{Q}}| \leq 2^{\beta_1}$ for each feasible
charge sector~$\mathbf{Q}$, with strict inequality whenever some cotree
assignments fail to extend.

\paragraph{All-even coordination (single sector).}
When every vertex has even degree, $\mathbf{Q} = \mathbf{0}$ is the
only feasible sector.  The per-sector bound is the global bound:
$|\calI| \leq 2^{\beta_1}$.  For the $3 \times 3$ periodic square
lattice ($\beta_1 = 10$): $2^{10} = 1024$, while brute-force enumeration
gives $|\calI| = 148$---barely $14\%$ of the topological bound.
The gap between $2^{\beta_1}$ and $|\calI|$ is precisely what the
directed-cycle constraint (\cref{sec:counting_table}) quantifies.

\paragraph{Mixed coordination (multiple sectors).}
When some vertices have odd degree, multiple charge sectors exist and the
total ice manifold sums across them:
\begin{equation}
  |\calI| = \sum_{\mathbf{Q} \in \calQ_{\min}} |\calI_{\mathbf{Q}}|
  \leq |\calQ_{\min}| \cdot 2^{\beta_1}.
  \label{eq:global_bound}
\end{equation}
In particular, $|\calI|$ can exceed $2^{\beta_1}$.
The counting table (\cref{tab:counting}) confirms this: the square
$4\!\times\!4$ open lattice has $\beta_1 = 9$, so $2^{\beta_1} = 512$,
but $|\calI| = 2{,}768$ across $|\calQ_{\min}| = 70$ charge sectors.
The spanning-tree argument itself is unchanged---at step~3, one
propagates $(B_1\bm\sigma)_v = Q_v$ instead of $(B_1\bm\sigma)_v = 0$,
which shifts the determined tree-edge values but does not alter the
counting logic.

\paragraph{Face-filling caveat.}
The bound requires $\beta_1$ with \emph{no faces filled}
($n_2 = 0$), where $\beta_1 = n_1 - n_0 + 1 = n_0 + 1$ grows extensively
for the periodic square lattice.  With all faces filled, $\beta_1 = 2$
(the two torus winding modes) regardless of system size, so
$2^{\beta_1} = 4$ is \emph{not} an upper bound on $|\calI|$---it gives
$s \to 0$ in the thermodynamic limit while $|\calI|$ grows exponentially.
The all-faces $\beta_1$ counts only the topologically nontrivial
(harmonic) dimensions of the cycle space, not its full size.

\subsection{Kinetics: single flips vs.\ loop flips}
\label{sec:kinetics}

\textbf{Single spin flip} (flipping one edge $e$): changes $Q_v \mapsto Q_v \pm 2$
at both endpoints. Moves the configuration to a \emph{different} Coulomb class.
Creates or annihilates a monopole-antimonopole pair.

\paragraph{Cycles and the cycle space.}
An \emph{oriented cycle} on $G$ is a vector
$\mathbf{c} \in \{-1, 0, +1\}^{n_1}$ supported on the edges of a closed
path, with a consistent algebraic orientation: every vertex on the path has
exactly one cycle edge with $[B_1]_{v,e}\,c_e = +1$ (entering) and one with
$[B_1]_{v,e'}\,c_{e'} = -1$ (leaving), so that
\begin{equation}
  (B_1 \mathbf{c})_v
  = \underbrace{(+1)}_{\text{head of one edge}}
  + \underbrace{(-1)}_{\text{tail of another}}
  = 0
  \label{eq:cycle_divergence}
\end{equation}
for every vertex $v$ on the cycle (and trivially zero for vertices off the
cycle).  Therefore every oriented cycle satisfies
$\mathbf{c} \in \ker(B_1)$.  This is a property of the cycle vector
$\mathbf{c}$ alone---it is independent of any spin
configuration~$\bm\sigma$.  More generally, any integer linear
combination of simple cycles also lies in $\ker(B_1)$; the full space
$\ker(B_1)$ is the \textbf{cycle space} of~$G$, with dimension
$n_1 - n_0 + 1$ for a connected graph (\cref{eq:dim_kerB1}).

\textbf{Loop flip} (negating all spins along a cycle
$\mathbf{c} \in \ker(B_1)$): $\sigma_e \mapsto -\sigma_e$ for every
$e \in \mathrm{supp}(\mathbf{c})$.  This preserves the ice rule
\emph{when the cycle is directed} in the current configuration
(\cref{sec:counting_table}).  In the additive picture
($\bm\sigma \mapsto \bm\sigma + 2\mathbf{c}$), charge preservation
follows immediately from $B_1\mathbf{c} = \mathbf{0}$; however, the
result stays in $\{-1,+1\}^{n_1}$ only when the cycle is directed,
in which case the additive and multiplicative flips coincide.
When both conditions hold, the configuration stays in the same Coulomb
class.  A classical theorem of Veblen~\cite{veblen1912} guarantees that
any two ice states in the same charge sector are connected by a
sequence of directed cycle reversals (\cref{thm:veblen_connectivity}).

\paragraph{Two kinds of cycle.}
The Hodge decomposition (\cref{sec:hodge}) refines the cycle space into
two orthogonal summands:
\begin{equation}
  \ker(B_1) = \im(B_2) \;\oplus\; \ker(L_1).
  \label{eq:cycle_hodge_preview}
\end{equation}
\emph{Exact cycles} ($\im B_2$) are boundaries of faces:
each column of $B_2$ is the oriented boundary of a face, and
$B_1 B_2 = 0$ guarantees these are cycles.  Flipping an exact cycle
rearranges spins around one or more faces---a purely local move.
\emph{Harmonic cycles} ($\ker L_1$) are topologically nontrivial: they
are divergence-free ($B_1\mathbf{c} = 0$) \emph{and} curl-free
($B_2^T\mathbf{c} = 0$), meaning they are not the boundary of any
collection of faces.  On a torus, the two harmonic modes correspond to net
circulation around the two non-contractible loops (horizontal and vertical
winding numbers); no sequence of face-boundary flips can change them.

\subsection{Loop-basis parameterization of the ice manifold}
\label{sec:loop_basis}

\begin{convention}[Face-Filling]
Throughout \cref{sec:loop_basis,sec:counting_table}, we use $n_2 = 0$
(no faces filled), so $\beta_1 = \dim(\ker B_1) = n_1 - n_0 + 1$.
\end{convention}

\subsubsection{\texorpdfstring{The $\bm\alpha$ parameterization}{The alpha parameterization}}

A basis $\{\mathbf{l}_1, \ldots, \mathbf{l}_{\beta_1}\}$ of $\ker(B_1)$
consists of $\beta_1$ independent oriented cycles with $B_1\mathbf{l}_i = \mathbf{0}$.
Given a seed ice state $\bm\sigma_{\text{seed}}$, each cycle defines a
potential \textbf{loop flip}: negating all spins along the cycle,
$\sigma_e \mapsto -\sigma_e$ for every $e \in \mathrm{supp}(\mathbf{l}_i)$.
A binary vector $\bm\alpha \in \{0,1\}^{\beta_1}$ selects which cycles to
flip, giving the parameterization
\begin{equation}
  \sigma_e(\bm\alpha) = \sigma_{\text{seed},e} \cdot (-1)^{%
    \sum_{i:\,e \in \text{supp}(\mathbf{l}_i)} \alpha_i}.
  \label{eq:alpha_param}
\end{equation}
This formula is order-independent: the resulting spin configuration depends
only on \emph{which} cycles are selected, not on the sequence in which they
are flipped.  Every $\bm\alpha$ yields a well-defined configuration in
$\{-1,+1\}^{n_1}$, but not every such configuration is an ice state.
Whether a cycle flip preserves the ice rule depends on whether the cycle
is \emph{directed} in the current spin configuration---a concept we now
illustrate before proceeding to the general theory.

\subsubsection{Worked example: directed vs.\ non-directed cycle flips}
\label{sec:directedness_example}

\Cref{fig:directedness_example} illustrates the key distinction on
the $4\!\times\!4$ open square lattice ($n_0 = 16$, $n_1 = 24$,
$\beta_1 = 9$).  Consider an ice state~$\bm\sigma$ and two
basis cycles, each a unit-cell plaquette.

\paragraph{Directed cycle (\cref{fig:directedness_example}a--b).}
The blue cycle visits vertices $8 \to 9 \to 13 \to 12 \to 8$,
forming a square plaquette.  In the current configuration~$\bm\sigma$,
the spin flow along this cycle is consistent: at every cycle
vertex, one cycle edge carries spin \emph{in} and one carries spin
\emph{out}.  The cycle is therefore \emph{directed}.
Flipping all four cycle edges (\cref{fig:directedness_example}b)
reverses the arrows but preserves the in/out balance at every vertex:
$\Delta Q_v = 0$ everywhere, and the result is a new ice state.

\paragraph{Non-directed cycle (\cref{fig:directedness_example}c--d).}
The red cycle visits vertices $4 \to 5 \to 9 \to 8 \to 4$,
the plaquette directly above.  In the \emph{same}
configuration~$\bm\sigma$, the spin flow is not consistent around
this cycle: at vertex~4, both cycle edges carry spin \emph{into}
the vertex (``both IN''), and at vertex~8, both carry spin
\emph{out} (``both OUT'').  The cycle is \emph{not directed}.
Flipping these four edges (\cref{fig:directedness_example}d)
reverses the two incoming edges at vertex~4 to outgoing,
changing the net flow by $\Delta Q_4 = -4$; symmetrically,
$\Delta Q_8 = +4$.  The result has $|Q_4| = 3$ and $|Q_8| = 3$---a
monopole--antimonopole pair---violating the ice rule.

Note that directedness is a property of the \emph{pair}
$(\text{cycle}, \bm\sigma)$, not of the cycle alone.
Flipping a directed cycle changes $\bm\sigma$ on the flipped edges,
which can alter the directedness of any other cycle that shares
edges with it: a cycle that was directed may become non-directed,
and vice versa.  In this example, the directed flip in panel~(b)
causes a neighboring cycle (sharing one edge) to lose its directedness.
This cascading interaction is what makes the decision tree
nontrivial (\cref{sec:counting_table}).

\begin{figure}[t]
  \centering
  \includegraphics[width=\columnwidth]{directedness_example}
  \caption{Directed vs.\ non-directed cycle flips on the $4\!\times\!4$
    open square lattice.
    \textbf{(a)}~An ice state~$\bm\sigma$ with a directed basis cycle
    (blue): at each cycle vertex, one cycle edge flows in and one flows out.
    \textbf{(b)}~After flipping the directed cycle, all arrows on the cycle
    reverse, but the in/out balance is preserved---still an ice state.
    \textbf{(c)}~The same~$\bm\sigma$ with a non-directed basis cycle
    (red): vertex~4 has both cycle edges flowing in (orange, ``both IN'');
    vertex~8 has both flowing out (``both OUT'').
    \textbf{(d)}~After flipping the non-directed cycle, vertices 4 and 8
    acquire charges $Q = -3$ and $Q = +3$ (red circles)---a
    monopole--antimonopole pair that violates the ice rule.}
  \label{fig:directedness_example}
\end{figure}

\subsubsection{The GF(2) bijection theorem}

We now show that the $\bm\alpha$ parameterization establishes a bijection
between ice states in a charge sector and a subset of
$\{0,1\}^{\beta_1}$.

\begin{theorem}[GF(2) Bijection]
\label{thm:gf2_bijection}
Let $G$ be a connected graph with fundamental cycle basis
$\{\mathbf{l}_1, \ldots, \mathbf{l}_{\beta_1}\}$ from a spanning tree~$T$.
Fix a charge vector $\mathbf{Q}$ and a seed ice state
$\bm\sigma_{\text{seed}}$ with $B_1\bm\sigma_{\text{seed}} = \mathbf{Q}$.
Define $\calI_{\mathbf{Q}} = \{\bm\sigma \in \{-1,+1\}^{n_1} : B_1\bm\sigma = \mathbf{Q}\}$
and $A_{\mathbf{Q}} = \{\bm\alpha \in \{0,1\}^{\beta_1} : \bm\sigma(\bm\alpha) \in \calI_{\mathbf{Q}}\}$.
Then:
\begin{enumerate}
  \item \emph{(Injectivity)} The map $\bm\alpha \mapsto \bm\sigma(\bm\alpha)$
    is injective: distinct $\bm\alpha$'s produce distinct configurations.
  \item \emph{(Surjectivity)} Every ice state
    $\bm\sigma' \in \calI_{\mathbf{Q}}$ has a unique
    $\bm\alpha \in \{0,1\}^{\beta_1}$ with $\bm\sigma(\bm\alpha) = \bm\sigma'$.
\end{enumerate}
Hence $|A_{\mathbf{Q}}| = |\calI_{\mathbf{Q}}|$.
\end{theorem}

\begin{proof}
\emph{Injectivity.}  The $\bm\alpha$ parameterization acts on
edges by $\mathrm{GF}(2)$ linear combination: the set of flipped edges is
$S(\bm\alpha) = \bigoplus_{i:\alpha_i=1} \mathrm{supp}(\mathbf{l}_i)$
(symmetric difference).  Since the fundamental cycle supports are linearly
independent over $\mathrm{GF}(2)$~\cite{whitney1932},
$S(\bm\alpha) = S(\bm\alpha')$ implies $\bm\alpha = \bm\alpha'$.
Distinct flip sets produce distinct configurations (they differ on at least
one edge).

\emph{Surjectivity.}  Given $\bm\sigma' \in \calI_{\mathbf{Q}}$, the
\emph{flip set} $S = \{e : \sigma'_e \neq \sigma_{\text{seed},e}\}$
satisfies: at every vertex~$v$, the number of edges in~$S$ that are incoming
at~$v$ (in either $\bm\sigma'$ or $\bm\sigma_{\text{seed}}$) equals the
number that are outgoing.  This is because
$B_1\bm\sigma' = B_1\bm\sigma_{\text{seed}} = \mathbf{Q}$, so flipping
$S$ preserves the charge at every vertex.  Therefore~$S$ has even degree at
every vertex, which means~$S$ (as an edge set) lies in the $\mathrm{GF}(2)$
cycle space of~$G$.  Since the fundamental cycle supports form a
$\mathrm{GF}(2)$ basis for this cycle space, $S$ decomposes uniquely as
$S = \bigoplus_{i \in J} \mathrm{supp}(\mathbf{l}_i)$ for some
$J \subseteq \{1, \ldots, \beta_1\}$.  Setting $\alpha_i = 1$ iff
$i \in J$ gives the unique preimage.
\end{proof}

\begin{corollary}
\label{cor:sector_bound}
$|\calI_{\mathbf{Q}}| = |A_{\mathbf{Q}}| \leq 2^{\beta_1}$.
\end{corollary}

This recovers the spanning-tree upper bound of \cref{sec:ice_manifold}
and makes it constructive: the $\bm\alpha$ map provides the explicit
bijection.

\subsubsection{Directed single-cycle flips}

A single-cycle flip negates $\sigma_e$ on the cycle's support, changing the
vertex charge at each cycle vertex~$v$ (where exactly two cycle edges meet)
by
\begin{equation}
  \Delta Q_v = -2\bigl([B_1]_{v,e_1}\,\sigma_{e_1}
               + [B_1]_{v,e_2}\,\sigma_{e_2}\bigr).
  \label{eq:deltaQ_preview}
\end{equation}
If the two cycle edges at $v$ carry spin in \emph{opposite} directions (one
flowing in, one flowing out), flipping both swaps their roles and
$\Delta Q_v = 0$: the in/out balance at $v$ is unchanged.  If instead both
carry spin in the \emph{same} direction, flipping both reverses them and
$\Delta Q_v = \pm 4$---an ice-rule violation.

A cycle is \emph{directed} in configuration~$\bm\sigma$ when the favorable
case holds at every vertex: opposite spin flow at each cycle vertex.
(The precise algebraic criterion appears in
\cref{sec:counting_table}.)  Directedness is a property of the
\emph{pair} $(\mathbf{l}_i, \bm\sigma)$, not of $\mathbf{l}_i$ alone:
the same cycle can be directed in one spin configuration and non-directed
in another.  A \emph{single} loop flip preserves the ice rule if and only
if the cycle is directed at the moment of the flip.

\subsubsection{\texorpdfstring{Why $|\calI_{\mathbf{Q}}| < 2^{\beta_1}$}{Why |I_Q| < 2^beta_1}}

Not every $\bm\alpha \in \{0,1\}^{\beta_1}$ produces an ice state.
The $\bm\alpha$ parameterization flips a set of edges
$S(\bm\alpha) = \bigoplus_{i:\alpha_i=1} \mathrm{supp}(\mathbf{l}_i)$
(the symmetric difference of selected basis cycle supports).  The result
$\bm\sigma(\bm\alpha)$ is an ice state if and only if the charge is
preserved at every vertex, i.e., the flip set~$S(\bm\alpha)$ has equal
numbers of in$\to$out and out$\to$in transitions at each vertex~$v$.

For a \emph{single} cycle ($|\{i : \alpha_i = 1\}| = 1$), this reduces to
the directedness condition: each cycle vertex sees exactly one incoming and
one outgoing cycle edge.  For \emph{multi-cycle} flips, the condition is
weaker: the \emph{effective} flip set $S(\bm\alpha)$---which may differ
from the union of individual cycle supports due to cancellations on shared
edges---must collectively preserve charge at every vertex.  Two individually
non-directed cycles can combine to produce a valid flip if their shared-edge
cancellations restore the charge balance.

\begin{keyresult}[$\bm\alpha$--Ice-State Bijection (Per Charge Sector)]
\label{kr:loop_basis}
Let $G$ be a connected graph with $n_2 = 0$, $\bm\sigma_{\text{seed}}$ an ice
state in charge sector $C_{\mathbf{Q}}$, and
$\{\mathbf{l}_1, \ldots, \mathbf{l}_{\beta_1}\}$ a basis of $\ker(B_1)$.
The $\bm\alpha$ parameterization~\eqref{eq:alpha_param} is a bijection
between $\calI_{\mathbf{Q}}$ (all ice states in the sector) and the subset
$A_{\mathbf{Q}} \subseteq \{0,1\}^{\beta_1}$ of charge-preserving
$\bm\alpha$ vectors (\cref{thm:gf2_bijection}).  This gives
\begin{equation}
  |\calI_{\mathbf{Q}}| = |A_{\mathbf{Q}}| \leq 2^{\beta_1}.
  \label{eq:loop_bound}
\end{equation}
This is a \textbf{per-sector bound}: it limits the number of
ice states within any single charge sector.  The total ice
manifold $|\calI| = \sum_{\mathbf{Q}} |\calI_{\mathbf{Q}}|$ sums over all
charge sectors and can exceed $2^{\beta_1}$ when multiple sectors exist.
\end{keyresult}

\noindent
The next section gives the precise algebraic definition of directedness,
derives the charge-preservation identity $\Delta Q_v = 0$ from
$B_1\mathbf{c} = \mathbf{0}$, and discusses the distinction between
the full ice manifold and the subset reachable by sequential basis-cycle
flips.

\subsection{The directed-cycle constraint and counting table}
\label{sec:counting_table}

\subsubsection{Alignment signs and the directedness criterion}

\textbf{Definition.} Let $\mathbf{c} \in \ker(B_1)$ be an oriented cycle
and $\bm\sigma \in \{-1,+1\}^{n_1}$ a spin configuration.  Define the
\emph{alignment sign} of each cycle edge $e \in \mathrm{supp}(\mathbf{c})$ as
\begin{equation}
  s_e \;=\; \sigma_e \, c_e \;\in\; \{-1, +1\}.
  \label{eq:alignment_sign}
\end{equation}
The cycle $\mathbf{c}$ is \emph{directed} in $\bm\sigma$ if the alignment
sign is constant across the cycle:
$s_e = s$ for every $e \in \mathrm{supp}(\mathbf{c})$, where $s = +1$ or
$s = -1$.  Equivalently, at every vertex on the cycle, exactly one cycle edge
carries spin into the vertex and one carries spin out (consistent
``spin current'' around the loop).

\emph{Note:} directedness is a property of the \emph{pair}
$(\mathbf{c}, \bm\sigma)$, not of $\mathbf{c}$ alone.  Every cycle
$\mathbf{c} \in \ker(B_1)$ is \emph{oriented} (has consistent algebraic
signs) by definition; directedness additionally requires the physical spins to
align with that orientation.  Changing $\bm\sigma$ can turn a directed cycle
into a non-directed one and vice versa.

\subsubsection{Charge preservation for directed single-cycle flips}

Under the multiplicative loop flip $\sigma_e' = -\sigma_e$ for
$e \in \mathrm{supp}(\mathbf{c})$ (with $\sigma_e' = \sigma_e$ off the
support), the vertex charge changes by
%
\begin{equation}
  \Delta Q_v
  \;=\; Q_v' - Q_v
  \;=\; -2 \!\!\sum_{e \in \mathrm{supp}(\mathbf{c})}
        [B_1]_{v,e}\,\sigma_e.
  \label{eq:deltaQ}
\end{equation}
%
Using $c_e^2 = 1$ on the support, write
$\sigma_e = s_e \cdot c_e$
where $s_e = \sigma_e\,c_e$ is the alignment sign
from~\eqref{eq:alignment_sign}.  Substituting into each term of the sum:
%
\begin{equation}
  [B_1]_{v,e}\,\sigma_e
  \;=\; s_e \cdot [B_1]_{v,e}\,c_e.
  \label{eq:factor_alignment}
\end{equation}
%
When $\mathbf{c}$ is directed in $\bm\sigma$, the alignment sign is constant
($s_e = s$ for all $e$), and $s$ factors out of the sum:
%
\begin{equation}
  \Delta Q_v
  \;=\; -2s \!\!\sum_{e \in \mathrm{supp}(\mathbf{c})}
        [B_1]_{v,e}\,c_e
  \;=\; -2s\,(B_1\mathbf{c})_v
  \;=\; 0,
  \label{eq:deltaQ_directed}
\end{equation}
%
where the last step uses $\mathbf{c} \in \ker(B_1)$.
The charge is preserved at every vertex, so the ice rule is maintained.
When $\mathbf{c}$ is \emph{not} directed, the signs $s_e$ vary across
edges and do not factor out; generically $\Delta Q_v \neq 0$, violating
the ice rule.

\subsubsection{Veblen connectivity of the ice manifold}
\label{sec:veblen}

A fundamental question is whether the ice manifold within a charge sector
is \emph{connected} under directed cycle moves.  The answer is yes, by a
classical theorem of Veblen~\cite{veblen1912}:

\begin{theorem}[Veblen Connectivity of Ice States]
\label{thm:veblen_connectivity}
Any two ice states $\bm\sigma, \bm\sigma'$ in the same charge sector
($B_1\bm\sigma = B_1\bm\sigma'$) are connected by a finite sequence of
directed cycle reversals.
\end{theorem}

\begin{proof}
Consider the flip set $S = \{e : \sigma'_e \neq \sigma_e\}$.  Because both
configurations have identical charge vectors, the subgraph induced by~$S$
has even degree at every vertex: each vertex sees equally many
in$\to$out and out$\to$in transitions.  Define an orientation on~$S$ by the
``new'' direction: edge~$e \in S$ is oriented according to~$\bm\sigma'$
(equivalently, opposite to~$\bm\sigma$ on these edges).  The resulting
oriented subgraph is balanced (equal in-degree and out-degree at every
vertex).  By Veblen's theorem~\cite{veblen1912}, every balanced digraph
decomposes into edge-disjoint directed cycles $C_1, \ldots, C_k$.  Since
the cycles are edge-disjoint, reversing $C_1$ in $\bm\sigma$ flips only
edges of~$C_1$; the remaining cycles $C_2, \ldots, C_k$ remain directed
in the resulting configuration.  Reversing all $k$ cycles sequentially
transforms $\bm\sigma$ into~$\bm\sigma'$.
\end{proof}

\paragraph{Critical consequence.}
The ice manifold within each charge sector is connected under
directed cycle moves---there are no ``disconnected islands'' of ice states.
Every ice state is reachable from every other in the same sector via
Veblen's decomposition.  (For related lattice-theoretic perspectives on
orientation spaces, see Propp~\cite{propp2002} and
Felsner~\cite{felsner2004}.)  The Veblen connecting cycles are, however,
\emph{arbitrary} directed cycles in the graph, not necessarily basis
cycles.  A single Veblen cycle may correspond to the
\emph{simultaneous} flip of multiple basis cycles.

\subsubsection{Sequential basis-cycle enumeration (the pruned decision tree)}
\label{sec:decision_tree}

Given a seed ice state $\bm\sigma_{\text{seed}}$, a cycle basis
$\{\mathbf{l}_1, \ldots, \mathbf{l}_{\beta_1}\}$, and a fixed ordering
(permutation) of the $\beta_1$ cycles, the \textbf{sequential basis-cycle
enumeration} explores a binary decision tree of depth~$\beta_1$ by the
following depth-first search (DFS):
\begin{enumerate}
  \item Initialize $\bm\sigma \leftarrow \bm\sigma_{\text{seed}}$ and
    set the current depth $k \leftarrow 1$.
  \item At depth~$k$, consider cycle $\mathbf{l}_{\pi(k)}$ (where $\pi$
    is the ordering).
    \begin{itemize}
      \item \textbf{No-flip branch} ($\alpha_{\pi(k)} = 0$): leave
        $\bm\sigma$ unchanged and recurse to depth $k+1$.
      \item \textbf{Flip branch} ($\alpha_{\pi(k)} = 1$): check whether
        $\mathbf{l}_{\pi(k)}$ is \emph{directed} in the current
        $\bm\sigma$ (\cref{eq:alignment_sign,eq:deltaQ_directed}).
        If yes, set $\sigma_e \leftarrow -\sigma_e$ for all
        $e \in \mathrm{supp}(\mathbf{l}_{\pi(k)})$ and recurse to
        depth $k+1$.  If no, \textbf{prune}: this branch is not explored
        (flipping would violate the ice rule).
    \end{itemize}
  \item At depth $\beta_1 + 1$, record $\bm\sigma$ as a reachable state.
\end{enumerate}
Each leaf of the unpruned tree corresponds to one
$\bm\alpha \in \{0,1\}^{\beta_1}$; pruning removes all leaves whose
path includes a non-directed flip, so the reachable set for a single
ordering $\pi$ is a subset of the ice manifold.

Because directedness depends on the current $\bm\sigma$---and flipping
one cycle can change the directedness of another
(\cref{sec:directedness_example})---different orderings prune different
branches and discover different states.  The \textbf{multi-ordering
reachable set} is
\begin{equation}
  \varepsilon_{\mathbf{Q}}
  = \bigcup_{k=1}^{K}
    \bigl\{\text{states found by DFS with ordering } \pi_k\bigr\},
  \label{eq:eps_def}
\end{equation}
where $\pi_1, \ldots, \pi_K$ are $K$ independent random permutations.
The count $|\varepsilon_{\mathbf{Q}}|$ grows monotonically with~$K$
but is bounded above by $|\calI_{\mathbf{Q}}|$; in practice it
saturates well before $K = 200$ for the small lattices in
\cref{tab:counting}.  Even in the exhaustive limit
$K = \beta_1!$ (every possible ordering), the union can remain strictly
smaller than $|\calI_{\mathbf{Q}}|$; we denote this exhaustive union
\begin{equation}
  \varepsilon_{\mathbf{Q}}^{*}
  = \bigcup_{\pi \in S_{\beta_1}}
    \bigl\{\text{states found by DFS with ordering } \pi\bigr\},
  \label{eq:eps_star_def}
\end{equation}
where $S_{\beta_1}$ is the symmetric group on $\beta_1$ elements.

\subsubsection{Four distinct quantities}
\label{sec:three_quantities}

The preceding results distinguish four quantities that must not be
conflated:
\begin{enumerate}
  \item $|\varepsilon_{\mathbf{Q}}|$: the number of states reachable by
    the sequential basis-cycle DFS of \cref{sec:decision_tree}
    (union over $K$ random orderings).
  \item $|\varepsilon_{\mathbf{Q}}^{*}|$: the exhaustive union over
    \emph{all} $\beta_1!$ orderings~\eqref{eq:eps_star_def}.
  \item $|\calI_{\mathbf{Q}}|$: the number of ice states in charge sector
    $\mathbf{Q}$.  By \cref{thm:gf2_bijection}, this equals the number of
    $\bm\alpha$ vectors that produce ice states: $|\calI_{\mathbf{Q}}| = |A_{\mathbf{Q}}|$.
  \item $2^{\beta_1}$: the total number of $\bm\alpha$ vectors
    (the topological upper bound from \cref{cor:sector_bound}).
\end{enumerate}

The correct ordering is:
\begin{equation}
  |\varepsilon_{\mathbf{Q}}|
  \;\leq\; |\varepsilon_{\mathbf{Q}}^{*}|
  \;\leq\; |\calI_{\mathbf{Q}}| = |A_{\mathbf{Q}}|
  \;\leq\; 2^{\beta_1}.
  \label{eq:four_hierarchy}
\end{equation}

\paragraph{Why $|\varepsilon_{\mathbf{Q}}^{*}| < |\calI_{\mathbf{Q}}|$.}
The sequential DFS (\cref{sec:decision_tree}) restricts to flipping
\emph{individual basis cycles} one at a time.  At each step, a cycle can
only be flipped if it is directed in the current $\bm\sigma$.  Some ice
states require \emph{simultaneous} multi-cycle flips: their $\bm\alpha$
vectors involve basis cycles that are individually non-directed from
$\bm\sigma_{\text{seed}}$---and remain non-directed after any sequence
of flips of the other directed basis cycles---yet whose combined effect
on the flip set $S(\bm\alpha)$ preserves charge at every vertex.  The
sequential DFS cannot reach these states regardless of ordering, because
at no point can it flip a non-directed basis cycle.

Exhaustive enumeration over all $\beta_1!$ orderings confirms this is
a \textbf{fundamental} limitation, not a sampling deficiency.
On the $3\!\times\!4$ open square ($\beta_1 = 6$), the exhaustive union
over all $6! = 720$ orderings yields
$|\varepsilon_{\mathbf{Q}}^{*}| = 10$, while brute-force enumeration
gives $|\calI_{\mathbf{Q}}| = 14$---a persistent gap of~4 states.
On the $5\!\times\!3$ open square ($\beta_1 = 8$), the exhaustive union
over all $8! = 40{,}320$ orderings yields
$|\varepsilon_{\mathbf{Q}}^{*}| = 27$ vs.\
$|\calI_{\mathbf{Q}}| = 31$.  In both cases, the missing states have
$\bm\alpha$ vectors requiring basis cycles (e.g., cycles~2 and~3 in the
$3\!\times\!4$ lattice) that are \emph{never} directed---not from the
seed, and not after flipping any combination of the remaining cycles.
By contrast, thin lattices ($2\!\times\!6$, $3\!\times\!3$, $4\!\times\!3$)
show $|\varepsilon_{\mathbf{Q}}^{*}| = |\calI_{\mathbf{Q}}|$: all
orderings together close the gap.

This is an \textbf{algorithmic limitation} of basis-restricted sequential
sampling, not a topological disconnection.  By
\cref{thm:veblen_connectivity}, all ice states in the sector are
connected by arbitrary directed cycle moves (not restricted to the
basis), so the missing states are topologically accessible---just not
via the pruned decision tree.

\paragraph{Basis dependence of the gap.}
The gap $|\varepsilon_{\mathbf{Q}}^{*}| < |\calI_{\mathbf{Q}}|$ depends
on the choice of spanning tree (and hence cycle basis), not on the ice
states themselves.  In fact, \emph{every} ice state is reachable by some
basis.  Given $\bm\sigma' \in \calI_{\mathbf{Q}}$, the flip set
$S = \{e : \sigma'_e \neq \sigma_{\text{seed},e}\}$ has even degree at
every vertex (both configurations share the same charge vector), so by
Veblen's theorem it decomposes into edge-disjoint directed cycles
$C_1, \ldots, C_k$.  Pick one edge $e_i$ from each~$C_i$; since each
$C_i$ is a cycle, removing $e_i$ does not disconnect the graph, so there
exists a spanning tree~$T$ avoiding all $e_1, \ldots, e_k$.  In the
fundamental cycle basis of~$T$, each $C_i$ is the fundamental cycle of
$e_i$.  Because the $C_i$ are edge-disjoint, flipping $C_1$ changes no
edges of $C_2, \ldots, C_k$---so all remain directed and can be flipped
in any order.  Hence the sequential DFS with basis~$T$ reaches
$\bm\sigma'$.  Computational verification on the $3\!\times\!4$ open
square confirms this: sampling over $500$ random spanning trees, some
bases achieve $|\varepsilon_{\mathbf{Q}}^{*}| = |\calI_{\mathbf{Q}}| = 14$
while others reach as few as~$1$ state.

\paragraph{Why different orderings find different states.}
Because directedness depends on $\bm\sigma$, and $\bm\sigma$ changes
after each flip, the pruned decision tree of
\cref{sec:decision_tree} is ordering-dependent: a cycle that is directed
at one point in the sequence may be non-directed at the same depth under
a different ordering.  This is why the multi-ordering union
$\varepsilon_{\mathbf{Q}}$~\eqref{eq:eps_def} discovers more states than
any single ordering.

\begin{keyresult}[Directed-Cycle Reduction and the Four-Quantity Hierarchy]
\label{kr:directed_cycle}
Within a single charge sector $C_{\mathbf{Q}}$, four quantities are related
by the strict hierarchy~\eqref{eq:four_hierarchy}:
$|\varepsilon_{\mathbf{Q}}| \leq |\varepsilon_{\mathbf{Q}}^{*}|
  \leq |\calI_{\mathbf{Q}}| = |A_{\mathbf{Q}}| \leq 2^{\beta_1}$.
The first inequality reflects incomplete sampling of orderings; it can
be closed by taking $K = \beta_1!$.
The second inequality is \textbf{basis-dependent}: for a fixed cycle basis,
even the exhaustive union over all orderings can be strictly smaller than
$|\calI_{\mathbf{Q}}|$, because some ice states require simultaneous
multi-cycle flips through basis cycles that are individually never
directed.  However, every ice state is reachable by \emph{some} basis:
by Veblen's theorem, the flip set decomposes into edge-disjoint directed
cycles, and a spanning tree can always be chosen so that these cycles are
fundamental.  Thus the gap
$|\varepsilon_{\mathbf{Q}}^{*}| < |\calI_{\mathbf{Q}}|$ reflects the
choice of basis, not a topological obstruction.
The third inequality reflects the constraint that not every $\bm\alpha$
produces an ice state (the effective flip must preserve charge at every
vertex).
The ratio $|\varepsilon_{\mathbf{Q}}|/2^{\beta_1}$ varies by five orders of
magnitude across the Morrison et al.\ lattice zoo (from 18\% on Santa Fe to
$6 \times 10^{-4}\%$ on tetris $3\times 3$).
\end{keyresult}

\subsubsection{Counting table}

\Cref{tab:counting} presents the central quantitative result: the full
counting table across eight lattice/size/BC combinations.

\begin{table*}[t]
\centering
\caption{Ice-state counting across the lattice zoo. $n_{\text{odd}}$:
number of odd-degree vertices. $|\calQ_{\min}|$: number of feasible charge
sectors. $|\calI_{\text{P}}|$: Pauling approximation.
$|\calI|$: independently verified ice-state count (= $|A_{\mathbf{Q}}|$ summed
over sectors, by \cref{thm:gf2_bijection}).
$|\varepsilon_{\mathbf{Q}}|$: states reachable by the sequential basis-cycle
DFS (\cref{sec:decision_tree}; $K = 200$ random orderings) within the seed's
charge sector---a lower bound on~$|\calI_{\mathbf{Q}}|$.}
\label{tab:counting}
\small
\begin{tabular}{lllrrcrrrrrr}
\toprule
\textbf{Lattice} & \textbf{Size} & \textbf{BC} & $n_1$ & $n_{\text{odd}}$
  & $|\calP|$ & $\beta_1$ & $2^{\beta_1}$
  & $|\calQ_{\min}|$ & $|\calI_{\text{P}}|$ & $|\calI|$
  & $|\varepsilon_{\mathbf{Q}}|$ \\
\midrule
Square  & $4\!\times\!4$ & open     & 24 &  8 & $1.7\!\times\!10^7$  &  9 & 512       &    70 & 2,076              & 2,768             & 38 \\
Square  & $4\!\times\!4$ & periodic & 32 &  0 & $4.3\!\times\!10^9$  & 17 & 131,072   &     1 & 657                & 2,970             & 299$^*$ \\
Kagome  & $2\!\times\!2$ & open     &  7 &  6 & 128                  &  0 & 1         &    20 & 18                 & 18                & 1 \\
Kagome  & $2\!\times\!2$ & periodic & 12 &  8 & 4,096                &  5 & 32        &    70 & 410                & 450               & 2 \\
Santa Fe & $2\!\times\!2$ & open    & 30 &  8 & $1.1\!\times\!10^9$  &  7 & 128       &    68 & 923                & 1,312             & 23 \\
Shakti  & $2\!\times\!2$ & open     & 81 & 26 & $2.4\!\times\!10^{24}$ & 18 & 262,144 & $\geq$1,881$^\ddagger$ & $1.6\!\times\!10^9$ & $>$1.4M$^\dagger$ & 248 \\
Tetris  & $2\!\times\!2$ & open     & 38 & 16 & $2.7\!\times\!10^{11}$ &  7 & 128     & 9,232 & 74,734             & 86,560            & 17 \\
Tetris  & $3\!\times\!3$ & open     & 93 & 36 & $9.9\!\times\!10^{27}$ & 22 & 4,194,304 & $\geq$1,984$^\ddagger$ & $1.9\!\times\!10^{12}$ & $>$3.1M$^\dagger$ & 25 \\
\bottomrule
\end{tabular}
\begin{flushleft}
\footnotesize
$^\dagger$ Lower bound only; backtracking search timed out at 120\,s.\\
$^\ddagger$ Lower bound from random sampling (2,000 ice states); exact
  enumeration infeasible.\\
$^*$ Lower bound; $K=200$ random orderings may not saturate.
\end{flushleft}
\end{table*}

\textbf{Charge sectors.}
Vertices with odd coordination number admit $Q_v = \pm 1$. The ice manifold splits
into charge sectors indexed by the $\pm 1$ pattern at the $n_{\text{odd}}$
odd-degree vertices, subject to $\sum_v Q_v = 0$. For lattices with all-even
bulk coordination (e.g., square with $z=4$), periodic BCs give
$n_{\text{odd}} = 0$ and a single sector; for odd-coordination lattices
(e.g., kagome with $z=3$), multiple sectors persist even with periodic BCs.
The sector count explains why $|\calI|$ can exceed $2^{\beta_1}$: the bound
applies per sector (\cref{kr:loop_basis}), but total states sum across all
$|\calQ_{\min}|$ sectors.

\textbf{Multi-ordering gains.}
Multi-ordering enumeration finds substantially more states than any single
ordering. For example: square $4\!\times\!4$ periodic finds 299 states
vs.\ 27--40 per single ordering ($|\calI| = 2{,}970$); tetris $3\!\times\!3$
open finds 25 vs.\ 4--6 (a $6\times$ increase).

\textbf{The reduction ratio} $|\varepsilon_{\mathbf{Q}}|/2^{\beta_1}$ ranges
from 18\% (Santa Fe $2\!\times\!2$ open, $23/128$) to
$6 \times 10^{-4}\%$ (tetris $3\!\times\!3$ open, $25/4{,}194{,}304$).
This five-order-of-magnitude range---invisible from $\beta_1$ alone---is a
fundamental structural property of the lattice.

\textbf{Computational verification.}
For lattices small enough to permit brute-force enumeration of all
$2^{n_1}$ configurations (square $4\!\times\!4$ open with $n_1 = 24$,
square $3\!\times\!3$ periodic with $n_1 = 18$), we have verified the
GF(2) bijection (\cref{thm:gf2_bijection}) directly: every ice state in
the seed's charge sector has a unique $\bm\alpha$, and
$|A_{\mathbf{Q}}| = |\calI_{\mathbf{Q}}|$ exactly.  We also confirm the
strict hierarchy~\eqref{eq:four_hierarchy}: for the $4\!\times\!4$ open
square, $|\varepsilon_{\mathbf{Q}}| = 38 < |\calI_{\mathbf{Q}}| = 54 < 2^9 = 512$;
for the $3\!\times\!3$ periodic square,
$|\varepsilon_{\mathbf{Q}}| = 101 < |\calI_{\mathbf{Q}}| = 148 < 2^{10} = 1024$.
Examining the 16 ``missing'' states in the open case reveals that their
$\bm\alpha$ vectors involve basis cycles that are individually non-directed
from the seed, consistent with the multi-cycle simultaneous-flip mechanism
described above.  Furthermore, exhaustive enumeration over all $\beta_1!$
orderings on smaller lattices confirms that the gap
$|\varepsilon_{\mathbf{Q}}^{*}| < |\calI_{\mathbf{Q}}|$ is fundamental:
for the $3\!\times\!4$ open square ($\beta_1 = 6$), all $720$ orderings
together yield only $10$ of $14$ ice states; for the $5\!\times\!3$ open
square ($\beta_1 = 8$), all $40{,}320$ orderings yield only $27$ of $31$.

% ===================================================================
% SECTION 6: HODGE DECOMPOSITION
% ===================================================================
\section{The Hodge Decomposition}
\label{sec:hodge}

\subsection{The Hodge decomposition theorem}

The edge signal space of a cell complex $X$ with boundary operators
$B_1$ and $B_2$ decomposes as
\begin{equation}
  \boxed{\R^{n_1} = \im(B_1^T) \;\oplus\; \im(B_2) \;\oplus\; \ker(L_1)}
  \label{eq:hodge}
\end{equation}
where the three subspaces are mutually orthogonal:
\begin{itemize}
  \item $\im(B_1^T)$: \textbf{gradient} signals---discrete gradients of
    vertex potentials;
  \item $\im(B_2)$: \textbf{curl} signals---boundaries of face signals;
  \item $\ker(L_1)$: \textbf{harmonic} signals---simultaneously
    divergence-free and curl-free.
\end{itemize}

\emph{Proof sketch.}
Since $L_1 = B_1^T B_1 + B_2 B_2^T$, the kernel satisfies
$\ker(L_1) = \ker(B_1) \cap \ker(B_2^T)$, because
$L_1\mathbf{f} = 0 \Leftrightarrow \|B_1\mathbf{f}\|^2 + \|B_2^T\mathbf{f}\|^2
= 0$. Orthogonality of $\im(B_1^T)$ and $\im(B_2)$ follows from the chain
complex property $B_1 B_2 = 0$: for $\mathbf{u} = B_1^T\mathbf{x}$ and
$\mathbf{v} = B_2\mathbf{y}$, we have
$\langle\mathbf{u},\mathbf{v}\rangle = \mathbf{x}^T B_1 B_2\mathbf{y} = 0$.
Completeness follows from the dimension count.

\subsection{The divergence-free subspace splits}

The divergence-free subspace $\ker(B_1)$---the ice manifold for even-degree
graphs, the fluctuation space for odd-degree graphs---decomposes as
\begin{equation}
  \boxed{\ker(B_1) = \im(B_2) \;\oplus\; \ker(L_1).}
  \label{eq:ice_hodge}
\end{equation}

\emph{Proof.}
$(\supseteq)$: $\im(B_2) \subseteq \ker(B_1)$ by $B_1 B_2 = 0$, and
$\ker(L_1) \subseteq \ker(B_1)$ since harmonic signals are divergence-free.
$(\subseteq)$: For $\mathbf{f} \in \ker(B_1)$, write
$\mathbf{f} = \mathbf{f}_{\text{grad}} + \mathbf{f}_{\text{curl}} +
\mathbf{f}_{\text{harm}}$ via~\eqref{eq:hodge}. Then
$0 = B_1\mathbf{f} = B_1\mathbf{f}_{\text{grad}}$, which forces
$\mathbf{f}_{\text{grad}} = B_1^T\mathbf{x}$ with $L_0\mathbf{x} = 0$;
for connected graphs, $\mathbf{x}$ is constant and
$\mathbf{f}_{\text{grad}} = \mathbf{0}$.~$\square$

\subsection{Dimensions and Betti numbers}

The Betti numbers are defined as
$\beta_k = \dim(\ker L_k)$, giving:
\begin{align}
  \dim(\im B_1^T) &= \rank(B_1) = n_0 - \beta_0, \\
  \dim(\im B_2)   &= \rank(B_2) = n_2 - \beta_2, \\
  \dim(\ker L_1)  &= \beta_1 = n_1 - \rank(B_1) - \rank(B_2).
  \label{eq:beta1}
\end{align}
The \textbf{Euler characteristic} provides a consistency check:
\begin{equation}
  \boxed{\beta_0 - \beta_1 + \beta_2 = n_0 - n_1 + n_2.}
  \label{eq:euler}
\end{equation}

\subsection{Physical interpretation}

\begin{table}[h]
\centering
\caption{Physical interpretation of Hodge components in spin ice.}
\label{tab:hodge_physical}
\begin{tabular}{llcp{5.5cm}}
\toprule
\textbf{Component} & \textbf{Subspace} & \textbf{Dim.} & \textbf{Spin Ice Meaning} \\
\midrule
Gradient  & $\im(B_1^T)$ & $n_0 - \beta_0$ &
  Monopole excitation modes; inter-class transitions \\
Curl      & $\im(B_2)$   & $n_2 - \beta_2$ &
  Circulations around faces; locally trivial ice modes \\
Harmonic  & $\ker(L_1)$  & $\beta_1$        &
  Topologically nontrivial winding modes; protected DoF \\
\bottomrule
\end{tabular}
\end{table}

On a torus (periodic square lattice), $\beta_1 = 2$: the two harmonic modes
correspond to net circulation around the two non-contractible loops. No sequence
of local plaquette flips can change these winding numbers.

\begin{keyresult}[Hodge Decomposition of Divergence-Free Subspace]
\label{kr:hodge}
\begin{equation}
  \ker(B_1) = \im(B_2) \oplus \ker(L_1).
\end{equation}
The divergence-free subspace decomposes into locally trivial circulations
($\im B_2$, dimension $n_2 - \beta_2$) and topologically protected winding
modes ($\ker L_1$, dimension $\beta_1$).

The left-hand side $\ker(B_1)$ depends only on the graph; its dimension is
always $n_1 - n_0 + \beta_0$. The decomposition of this fixed space into
the two summands depends on which faces are filled:
\begin{itemize}
  \item \emph{All faces filled:} $\im(B_2)$ maximal, $\beta_1$ minimal.
  \item \emph{No faces filled:} $B_2 = 0$, so $\ker(L_1) = \ker(B_1)$
    entirely and $\beta_1 = n_1 - n_0 + \beta_0$.
\end{itemize}
\end{keyresult}

\subsection{Laplacian definitions and topological invariants across the zoo}
\label{sec:topology_zoo}

To see how the Hodge machinery connects the boundary operators of
\cref{sec:boundary_ops} to the Betti numbers of the preceding subsections,
we collect all three Laplacians in one place.
The \textbf{graph (vertex) Laplacian} acts on vertex signals:
\begin{equation}
  \boxed{L_0 = B_1\, B_1^T,}
  \qquad (n_0 \times n_0), \qquad
  \dim\ker L_0 = \beta_0.
  \label{eq:L0_def}
\end{equation}
The \textbf{Hodge 1-Laplacian} acts on edge signals and splits into
lower and upper parts:
\begin{equation}
  \boxed{L_1 = \underbrace{B_1^T B_1}_{L_1^{\text{down}}}
             + \underbrace{B_2\, B_2^T}_{L_1^{\text{up}}},}
  \qquad (n_1 \times n_1), \qquad
  \dim\ker L_1 = \beta_1.
  \label{eq:L1_def}
\end{equation}
The \textbf{face Laplacian} acts on face signals:
\begin{equation}
  \boxed{L_2 = B_2^T B_2,}
  \qquad (n_2 \times n_2), \qquad
  \dim\ker L_2 = \beta_2.
  \label{eq:L2_def}
\end{equation}
All three are symmetric positive semi-definite, and their null-space
dimensions are precisely the Betti numbers $\beta_0$, $\beta_1$, $\beta_2$
constrained by the Euler relation~\eqref{eq:euler}.

\Cref{tab:topology_zoo} tabulates the full simplex counts and Betti numbers
for every lattice in \cref{tab:counting} under both face-filling strategies
and both boundary conditions, verifying these identities concretely.

\begin{table*}[t]
\centering
\caption{Topological invariants across the lattice zoo under both
face-filling strategies.  $n_2$ and $\chi = n_0 - n_1 + n_2$ refer to the
all-faces-filled complex.  Betti numbers are shown for both strategies:
``none'' ($n_2 = 0$) and ``all'' (all minimal faces filled).
$\beta_0 = 1$ throughout (connected graphs).
Consistency checks: $\chi = \beta_0 - \beta_1(\text{all}) + \beta_2(\text{all})$
and $\beta_1(\text{none}) = n_1 - n_0 + 1$.}
\label{tab:topology_zoo}
\small
\begin{tabular}{lllrrrrccccc}
\toprule
\textbf{Lattice} & \textbf{Size} & \textbf{BC}
  & $n_0$ & $n_1$ & $n_2$ & $\chi$
  & $\beta_0$ & $\beta_1(\text{none})$ & $\beta_1(\text{all})$
  & $\beta_2(\text{none})$ & $\beta_2(\text{all})$ \\
\midrule
Square   & $4\!\times\!4$ & open     &  16 &  24 &   9 &  1 & 1 &   9 &  0 & 0 & 0 \\
Square   & $4\!\times\!4$ & periodic &  16 &  32 &  16 &  0 & 1 &  17 &  2 & 0 & 1 \\
Square   & $6\!\times\!6$ & open     &  36 &  60 &  25 &  1 & 1 &  25 &  0 & 0 & 0 \\
Square   & $6\!\times\!6$ & periodic &  36 &  72 &  36 &  0 & 1 &  37 &  2 & 0 & 1 \\
\midrule
Kagome   & $2\!\times\!2$ & open     &   8 &   7 &   0 &  1 & 1 &   0 &  0 & 0 & 0 \\
Kagome   & $2\!\times\!2$ & periodic &   8 &  12 &   4 &  0 & 1 &   5 &  2 & 0 & 1 \\
Kagome   & $3\!\times\!3$ & open     &  18 &  19 &   2 &  1 & 1 &   2 &  0 & 0 & 0 \\
Kagome   & $3\!\times\!3$ & periodic &  18 &  27 &   9 &  0 & 1 &  10 &  2 & 0 & 1 \\
\midrule
Santa Fe & $2\!\times\!2$ & open     &  24 &  30 &   7 &  1 & 1 &   7 &  0 & 0 & 0 \\
Santa Fe & $2\!\times\!2$ & periodic &  24 &  36 &  12 &  0 & 1 &  13 &  2 & 0 & 1 \\
Santa Fe & $3\!\times\!3$ & open     &  54 &  72 &  19 &  1 & 1 &  19 &  0 & 0 & 0 \\
Santa Fe & $3\!\times\!3$ & periodic &  54 &  81 &  27 &  0 & 1 &  28 &  2 & 0 & 1 \\
\midrule
Shakti   & $2\!\times\!2$ & open     &  64 &  81 &  18 &  1 & 1 &  18 &  0 & 0 & 0 \\
Shakti   & $2\!\times\!2$ & periodic &  64 &  96 &  32 &  0 & 1 &  33 &  2 & 0 & 1 \\
Shakti   & $3\!\times\!3$ & open     & 144 & 193 &  50 &  1 & 1 &  50 &  0 & 0 & 0 \\
Shakti   & $3\!\times\!3$ & periodic & 144 & 216 &  72 &  0 & 1 &  73 &  2 & 0 & 1 \\
\midrule
Tetris   & $2\!\times\!2$ & open     &  32 &  38 &   7 &  1 & 1 &   7 &  0 & 0 & 0 \\
Tetris   & $2\!\times\!2$ & periodic &  32 &  48 &  16 &  0 & 1 &  17 &  2 & 0 & 1 \\
Tetris   & $3\!\times\!3$ & open     &  72 &  93 &  22 &  1 & 1 &  22 &  0 & 0 & 0 \\
Tetris   & $3\!\times\!3$ & periodic &  72 & 108 &  36 &  0 & 1 &  37 &  2 & 0 & 1 \\
\bottomrule
\end{tabular}
\end{table*}

Several patterns emerge.  Reading across the Betti columns:
$\beta_0 = 1$ always (the graph is connected: every pair of vertices is
linked by a path, so $\ker(L_0)$ is spanned by the constant vector).
$\beta_2(\text{none}) = 0$ always (with no faces, $B_2$ is the empty
matrix, so there are no 2-chains and hence no 2-cycles).
$\beta_2(\text{all}) = 0$ for open BCs and $1$ for periodic.
The periodic case has $\beta_2 = 1$ because the faces tile a closed
surface (the torus), so the signed sum of all consistently oriented
faces has no boundary---$B_2^T(\text{all faces}) = \mathbf{0}$---giving
one independent 2-cycle in $\ker(L_2)$.  On an open lattice, the faces
do not close up: the boundary of the total face sum is nonzero (it
equals the outer boundary of the patch), so there is no 2-cycle and
$\beta_2 = 0$.  In short: $\beta_0$ counts connected components,
$\beta_1$ counts independent non-bounding loops, and $\beta_2$ counts
enclosed voids---exactly the intuition from classical homology.
$\beta_1(\text{none}) = n_1 - n_0 + 1$ regardless of boundary conditions,
because without faces the upper Laplacian vanishes ($L_1^{\text{up}} = 0$)
and $\ker(L_1) = \ker(B_1^\top)$.
$\beta_1(\text{all}) = 0$ for open BCs (every cycle on a planar patch
bounds a union of faces) and $2$ for periodic (the two non-contractible
winding modes of $H_1(\mathbb{T}^2) \cong \Z^2$).
Periodic BCs add wrap-around edges without adding vertices, so
$\beta_1(\text{none})$ roughly doubles relative to the open case.

% ===================================================================
% SECTION 7: HAMILTONIAN
% ===================================================================
\section{Spin Ice Hamiltonian as Quadratic Form in \texorpdfstring{$L_1^{\text{down}}$}{L1down}}
\label{sec:hamiltonian}

\subsection{The dumbbell Hamiltonian}

In the dumbbell model (Castelnovo et al.\ 2008, adapted by
Nisoli~\cite{nisoli2020}), the nearest-neighbor charge Hamiltonian is
\begin{equation}
  \calH[\bm\sigma] = \frac{\epsilon}{2}\sum_v Q_v^2,
  \label{eq:H_charge}
\end{equation}
where $\epsilon > 0$ is the energy scale and $Q_v = (B_1\bm\sigma)_v$.

\subsection{Derivation}

Writing the sum of squared charges as a squared norm:
\begin{equation}
  \sum_v Q_v^2 = \|B_1\bm\sigma\|^2
  = \bm\sigma^T B_1^T B_1\,\bm\sigma
  = \bm\sigma^T L_1^{\text{down}}\,\bm\sigma.
\end{equation}

\begin{keyresult}[Hamiltonian as $L_1^{\text{down}}$ Quadratic Form]
\label{kr:hamiltonian}
\begin{equation}
  \boxed{\calH[\bm\sigma] = \frac{\epsilon}{2}\,\bm\sigma^T
  L_1^{\text{down}}\,\bm\sigma = \frac{\epsilon}{2}\|B_1\bm\sigma\|^2}
  \label{eq:H_quadratic}
\end{equation}
The spin ice energy is a quadratic form in the lower Hodge Laplacian. This
is exact, not an approximation, connecting spin ice statistical mechanics
directly to the spectral theory of $L_1^{\text{down}}$.
\end{keyresult}

\subsection{Spectral consequences}

Let $\{\lambda_i, \mathbf{u}_i\}_{i=1}^{n_1}$ be the eigendecomposition of
$L_1^{\text{down}}$. Expanding $\bm\sigma = \sum_i c_i\mathbf{u}_i$:
\begin{equation}
  \calH = \frac{\epsilon}{2}\sum_i \lambda_i\,c_i^2.
  \label{eq:H_spectral}
\end{equation}

\begin{table}[h]
\centering
\caption{Energetic structure from the spectrum of $L_1^{\text{down}}$.}
\label{tab:spectral_energy}
\begin{tabular}{llp{5cm}}
\toprule
\textbf{Eigenvalue} & \textbf{Mode type} & \textbf{Energy contribution} \\
\midrule
$\lambda_i = 0$ & $\mathbf{u}_i \in \ker(B_1)$ &
  Zero---charge-free modes \\
$\lambda_i > 0$ & $\mathbf{u}_i \in \im(B_1^T)$ &
  $(\epsilon/2)\lambda_i c_i^2 > 0$---excited (charged) modes \\
\bottomrule
\end{tabular}
\end{table}

\textbf{Ground state manifold} $= \ker(L_1^{\text{down}}) = \ker(B_1)$:
configurations with zero energy are exactly the divergence-free configurations.
For even-degree graphs, these are the ice-rule states. For odd-degree graphs,
zero energy is unachievable by Ising spins; the ice manifold minimizes
$\calH$ at $\calH_{\min} = (\epsilon/2)\sum_v (z_v \bmod 2)$.

\textbf{Spectral gap} $\Delta$: the smallest nonzero eigenvalue of
$L_1^{\text{down}}$ sets the minimum energy cost of a monopole excitation:
\begin{equation}
  E_{\text{min\,monopole}} = \frac{\epsilon}{2}\,\Delta \cdot
  \min_{c_i \neq 0} c_i^2.
  \label{eq:spectral_gap}
\end{equation}
Lattices with larger spectral gaps have more robust ice rules.

\subsection{Nisoli's interaction matrix}

Nisoli writes the energy as $\calH = (\epsilon/2)\,\mathbf{Q}^T\mathbf{Q}$
for the nearest-neighbor case. The chain of equalities
\begin{equation}
  \frac{\epsilon}{2}\sum_v Q_v^2
  = \frac{\epsilon}{2}\|B_1\bm\sigma\|^2
  = \frac{\epsilon}{2}\,\bm\sigma^T L_1^{\text{down}}\bm\sigma
\end{equation}
shows the dumbbell Hamiltonian, the 1-cochain norm, and the Hodge quadratic
form are three expressions for the same quantity. Nisoli extends this to
long-range interactions via the graph Green's function
$L_0^{-1}$~\cite{nisoli2020}, but the nearest-neighbor version already
establishes the $L_1^{\text{down}}$ connection.

\subsection{Topological current: the face-level dual of topological charge}
\label{sec:topological_current}

The dumbbell Hamiltonian penalizes topological charge
$Q_v = (B_1 \bm\sigma)_v$, which measures the net imbalance of incoming
versus outgoing spins at each vertex. As elementary electric charges
source an electric field, the natural question arises: is there
a dual quantity---a \emph{topological current}---that plays the same
role for faces that $Q_v$ plays for vertices?

Define the \textbf{topological current} on face $f$:
\begin{equation}
  J_f \;=\; (B_2^T \bm\sigma)_f
      \;=\; \sum_{e \in \partial f} \pm\, \sigma_e\,,
  \label{eq:topological_current}
\end{equation}
the signed circulation of spin around the boundary of $f$. Just as $Q_v$
measures how much spin \emph{diverges} from a vertex, $J_f$ measures how
much spin \emph{curls} around a face. When all edge spins flow
consistently around a face (a circulating current loop), $|J_f|$ is
maximized; when they cancel, $J_f = 0$.

The parallel between charge and current is exact
(Table~\ref{tab:charge_current}).

\begin{table}[h]
\centering
\caption{Charge--current duality for edge spins on a cell complex.
Each row shows the vertex-level (0-cell) quantity alongside its face-level
(2-cell) dual. The spin configuration $\bm\sigma$ plays the role of a
discrete vector potential; its divergence yields charge, its curl yields
current.}
\label{tab:charge_current}
\begin{tabular}{lcc}
\toprule
 & \textbf{Charge} ($Q_v$) & \textbf{Current} ($J_f$) \\
\midrule
Lives on       & vertices (0-cells) & faces (2-cells) \\
Observable     & $Q_v = (B_1\bm\sigma)_v$ & $J_f = (B_2^T\bm\sigma)_f$ \\
Operator       & $B_1$ (divergence) & $B_2^T$ (curl) \\
Laplacian      & $L_1^{\text{down}} = B_1^T B_1$
               & $L_1^{\text{up}} = B_2 B_2^T$ \\
Constraint     & Ice rule: $Q_v = 0$ & Current rule: $J_f = 0$ \\
Excitation     & Monopole   & Vortex \\
EM analogue    & Electric charge & Current loop / magnetic flux \\
\bottomrule
\end{tabular}
\end{table}

The spin configuration $\bm\sigma$ plays the role of a discrete vector
potential $\mathbf{A}$; the charge $Q_v = (\text{div}\;\bm\sigma)_v$ is its
divergence, and the current $J_f = (\text{curl}\;\bm\sigma)_f$ is its curl.

\subsection{The Hodge Hamiltonian}
\label{sec:hodge_hamiltonian}

With the current $J_f$ in hand, the Hamiltonian that penalizes both
charge and current excitations writes itself:
\begin{equation}
  \boxed{\;
  \mathcal{H}_{\text{Hodge}}
  = \frac{\epsilon}{2}\sum_v Q_v^2
  + \frac{\kappa}{2}\sum_f J_f^2
  = \frac{\epsilon}{2}\,\bm\sigma^T L_1^{\text{down}}\bm\sigma
  + \frac{\kappa}{2}\,\bm\sigma^T L_1^{\text{up}}\bm\sigma
  \;}
  \label{eq:hodge_hamiltonian}
\end{equation}
The coupling $\epsilon$ sets the energy cost of monopole excitations
(nonzero vertex charge), while $\kappa$ sets the cost of vortex
excitations (nonzero face current). The ratio $\kappa/\epsilon$
interpolates between pure spin-ice physics ($\kappa = 0$, only charge
matters) and pure plaquette physics ($\epsilon = 0$, only current
matters).

The connection to the Hodge decomposition is immediate. The constraint
$Q_v = 0\;\forall v$ (charge-free) restricts $\bm\sigma$ to
$\ker(B_1) = \operatorname{im}(B_2) \oplus \ker(L_1)$---the curl and
harmonic subspaces. The additional constraint
$J_f = 0\;\forall f$ (current-free) further restricts to
$\ker(B_2^T) \cap \ker(B_1) = \ker(L_1)$---the harmonic subspace alone.
When both couplings are nonzero, the ground-state manifold is precisely
$\ker(L_1)$, the space of configurations that are simultaneously
divergence-free and curl-free.

\subsection{Current frustration on odd-sided faces}
\label{sec:current_frustration}

There is a striking parallel between vertex frustration and what we
might call \emph{current frustration}. Since each $\sigma_e = \pm 1$,
the current $J_f$ is a sum of $|\partial f|$ terms each equal to $\pm 1$.
For a face with an \emph{odd} number of edges, $J_f$ is necessarily odd
and therefore $|J_f| \geq 1$: the current rule $J_f = 0$ can never be
satisfied. This is the face-level analogue of the familiar fact that
odd-coordinated vertices cannot satisfy the ice rule $Q_v = 0$.

This has immediate consequences for the lattice zoo:
\begin{itemize}
\item \textbf{Kagome} (all hexagonal faces): can satisfy
  $J_f = 0$ on every face (6 edges per face, even).
\item \textbf{Square lattice} (all square faces): can satisfy
  $J_f = 0$ on every face.
\item \textbf{Mixed lattices} (shakti, tetris, santa fe): current
  frustration depends on the mix of face sizes in the tiling.
\end{itemize}

\noindent
The $\kappa$ term in the Hodge Hamiltonian therefore has a nonzero
ground-state energy on lattices with odd-sided faces, just as the
$\epsilon$ term has a nonzero ground-state energy on lattices with
odd-coordinated vertices. For lattices that are frustrated in both
sectors, the harmonic subspace $\ker(L_1)$ may be empty, and the
system must compromise between charge and current costs.

\subsection{Physical realizations of the current term}
\label{sec:current_physics}

The charge--current decomposition and the Hodge Hamiltonian appear,
under various names, in several areas of physics:

\paragraph{Lattice gauge theory.}
In Wilson's lattice gauge theory~\cite{wilson1974}, the plaquette action
$S_{\text{plaq}} = -\beta \sum_f \cos J_f$ penalizes nonzero
circulation around faces, directly analogous to the $\kappa$ term for
Ising ($\mathbb{Z}_2$) gauge fields. The charge term plays the role of
Gauss's law. The full Hodge Hamiltonian thus combines gauge constraint
enforcement (charge) with dynamical content (current/flux).

\paragraph{Kitaev's toric code.}
The toric code Hamiltonian
$H_{\text{TC}} = -J_s \sum_v A_v - J_p \sum_f B_f$
is the quantum $\mathbb{Z}_2$ version of Eq.~\eqref{eq:hodge_hamiltonian}:
the \emph{star operators} $A_v$ enforce zero vertex charge (analogous to
$L_1^{\text{down}}$), while the \emph{plaquette operators} $B_f$ enforce
zero face current (analogous to $L_1^{\text{up}}$). Excitations of the
star term are electric charges ($e$-particles); excitations of the
plaquette term are magnetic vortices ($m$-particles). Both are anyons,
and their braiding statistics underlie topological quantum error
correction.

\paragraph{Discrete electromagnetism.}
On a cell complex, $Q_v = (B_1\bm\sigma)_v$ is the discrete
divergence and $J_f = (B_2^T\bm\sigma)_f$ is the discrete curl. The
Hodge Hamiltonian becomes the discrete analogue of electromagnetic
field energy
$\mathcal{E} \sim \int(\|\mathbf{E}\|^2 + \|\mathbf{B}\|^2)\,dV$,
with $\epsilon$ controlling the electric sector and $\kappa$ the
magnetic sector. Harmonic configurations ($\ker L_1$) are the discrete
analogue of source-free, curl-free fields---solutions to the vacuum
Maxwell equations.

\paragraph{Ring-exchange interactions.}
In quantum dimer models and pyrochlore spin ice, ring-exchange processes
flip spins around elementary plaquettes. The energy cost of such a flip
is controlled by $J_f$---precisely the face current. Hermele,
Fisher, and Balents showed that ring exchange in pyrochlore ice generates
an emergent $U(1)$ gauge field whose low-energy physics is governed by
the Maxwell action, the continuum limit of the Hodge
Hamiltonian~\cite{hermele2004}.

\paragraph{Dipolar spin ice.}
Real spin-ice materials (Dy$_2$Ti$_2$O$_7$, Ho$_2$Ti$_2$O$_7$)
have long-range dipolar interactions beyond the nearest-neighbor
dumbbell model. Castelnovo, Moessner, and Sondhi showed that these
corrections generate effective multi-spin interactions around
plaquettes~\cite{castelnovo2008}, introducing an energetic preference
among ice-rule-satisfying states that is naturally captured by the
$\kappa$ term: dipolar corrections lift the ice-manifold degeneracy
by penalizing configurations with large $|J_f|$.

\subsection{Implications for topological deep learning}
\label{sec:hodge_tdl}

The charge--current framework provides a physical interpretation of why
simplicial neural networks that use the full Hodge Laplacian $L_1$
access richer structure than graph neural networks using $L_0$ alone.
The $\epsilon$ term (graph Laplacian smoothing) drives signals toward
the charge-free subspace; the $\kappa$ term (upper Laplacian smoothing)
drives signals toward the current-free subspace. Only in
$\ker(L_1)$---signals that are both charge-free and current-free---do
features survive both smoothing channels indefinitely. The dimension
of this protected subspace is $\beta_1$, explaining its role as the
key topological invariant for oversmoothing resistance~\cite{li2018deeper}.

For lattice design, this suggests a new axis of control: beyond
maximizing $\beta_1$ (large harmonic subspace), one can separately tune
vertex frustration ($\epsilon$ sector) and current frustration ($\kappa$
sector) by choosing lattice topologies with specific mixes of vertex
coordination and face size.

% ===================================================================
% SECTION 8: OUTLOOK
% ===================================================================
\section{Outlook: Neural Sampling Approaches (Preliminary)}
\label{sec:outlook}

The TDL--ASI correspondence developed above suggests that it might be
possible to build neural samplers for spin ice configurations by adapting
recent autoregressive network architectures.  What follows is a preliminary
sketch of two such architectures; the ideas are concrete enough to write
down layer equations and training objectives, but we have not yet tested
them beyond small validation cases (Mode~A on square $4\times 4$).  There
are likely important details we are missing.

The design draws on three key developments.

Wu et al.~\cite{wu2019variational} introduced variational autoregressive
networks (VANs) for statistical mechanics: the joint distribution over $n$
binary spins is factored as
$q_\theta(\bm\sigma) = \prod_{i=1}^{n} q_\theta(\sigma_i \mid \sigma_{<i})$,
with each conditional parameterized by a neural network. Samples are drawn
in a single forward pass with tractable log-probabilities, enabling direct
gradient-based minimization of the variational free energy
$F_\theta = \langle \calH \rangle_{q_\theta}
+ T\,\langle \ln q_\theta \rangle_{q_\theta}$
via REINFORCE. Ma et al.~\cite{ma2024mpvan} extended this framework to the
MPVAN architecture, which incorporates the Hamiltonian coupling structure
into the message-passing mechanism. Their central finding is that using the
Hamiltonian couplings $J_{ij}$ as message weights outperforms all other
strategies---including learned weights and generic GCN
aggregation---on frustrated Ising models such as the $J_1$--$J_2$
antiferromagnet and the Sherrington--Kirkpatrick spin glass.

However, both VAN and MPVAN operate on \emph{vertex-level} spins (Ising
variables on nodes, coupled through edges), while ASI has \emph{edge-level}
spins (binary variables on edges, coupled through shared vertices). The EIGN
architecture of Fuchsgruber et al.~\cite{fuchsgruber2025} provides exactly
the missing piece: edge-level message passing that respects orientation
equivariance, using four operator channels built from $B_1$ and $|B_1|$.
The channels separate into self-modal (equ$\to$equ via $B_1^T B_1$ and
inv$\to$inv via $|B_1|^T|B_1|$) and cross-modal (equ$\to$inv and
inv$\to$equ) interactions. The key insight linking all three developments
is that EIGN's equivariant Laplacian $L_{\text{equ}} = B_1^T B_1$ is
\emph{identically} the ASI Hamiltonian $L_1^{\text{down}}$
(\cref{kr:hamiltonian}). MPVAN's principle of ``Hamiltonian-informed
message passing'' therefore holds \emph{by construction} when EIGN
operators are used---no separate Hamiltonian encoding step is needed.
This synthesis motivates two complementary sampling modes, both built on a
shared EIGN message-passing backbone.

\subsection{Shared architecture: the EIGN layer}
\label{sec:eign_layer}

Both modes use a stack of $K$ dual-channel EIGN layers that update
equivariant (spin-like) and invariant (geometry-like) edge features in
parallel. Writing
$X_{\text{equ}}^{(\ell)},\, X_{\text{inv}}^{(\ell)} \in \mathbb{R}^{n_1\times d}$
for the hidden representations at layer~$\ell$, the update rules are
\begin{align}
  X_{\text{equ}}^{(\ell+1)} &= \phi\!\Bigl(
    \underbrace{L_1^{\text{down}}\,X_{\text{equ}}^{(\ell)}\,W_1^{(\ell)}}_{\text{equ}\to\text{equ}}
    + \underbrace{B_1^T|B_1|\,X_{\text{inv}}^{(\ell)}\,W_2^{(\ell)}}_{\text{inv}\to\text{equ}}
    + X_{\text{equ}}^{(\ell)}\,W_5^{(\ell)}
  \Bigr),
  \label{eq:eign_equ} \\[4pt]
  X_{\text{inv}}^{(\ell+1)} &= \phi\!\Bigl(
    \underbrace{|B_1|^T|B_1|\,X_{\text{inv}}^{(\ell)}\,W_3^{(\ell)}}_{\text{inv}\to\text{inv}}
    + \underbrace{|B_1|^T B_1\,X_{\text{equ}}^{(\ell)}\,W_4^{(\ell)}}_{\text{equ}\to\text{inv}}
    + X_{\text{inv}}^{(\ell)}\,W_6^{(\ell)}
  \Bigr),
  \label{eq:eign_inv}
\end{align}
where $\phi$ is a nonlinear activation and
$\{W_k^{(\ell)}\}_{k=1}^{6}$ are learned weight matrices.  Each layer has
four message-passing channels plus two skip connections ($W_5, W_6$).  The
skip connections are essential: because ice-rule configurations lie in
$\ker(L_1^{\text{down}})$, the equ$\to$equ channel
$L_1^{\text{down}} X_{\text{equ}}$ annihilates all ice-state signal
components.  Without the skip path $X_{\text{equ}} W_5$, the network would
be blind to the very manifold it must learn to sample.

The equivariant input $X_{\text{equ}}^{(0)}$ is initialized from the
current spin configuration~$\bm\sigma$; the invariant input
$X_{\text{inv}}^{(0)}$ encodes static lattice geometry (vertex
coordination numbers, edge lengths, coupling constants).  After $K$ layers,
the final equivariant features are passed to a task-specific output head
described below.

\subsection{Mode A: LoopMPVAN (loop-basis autoregressive sampling)}
\label{sec:modeA}

Mode~A exploits the loop-basis parameterization of the ice manifold
(\cref{sec:loop_basis}).  Given a seed ice state $\bm\sigma_{\text{seed}}$
and $\beta_1$ independent loops $\{\bm l_i\}$ spanning $\ker(B_1)$, every
reachable ice state is encoded by a binary vector
$\bm\alpha \in \{0,1\}^{\beta_1}$, where $\alpha_i = 1$ means ``flip
loop~$i$.''  Mode~A autoregressively samples these loop-flip decisions:
\begin{equation}
  q_\theta(\bm\alpha)
  = \prod_{i=1}^{\beta_1}
    q_\theta\!\bigl(\alpha_i \mid \alpha_1,\ldots,\alpha_{i-1}\bigr).
  \label{eq:modeA_factor}
\end{equation}

\paragraph{Sampling procedure.}
At step~$i$, the current configuration
$\bm\sigma^{(i)} = \bm\sigma_{\text{seed}}
  \oplus \alpha_1\bm l_1 \oplus \cdots \oplus \alpha_{i-1}\bm l_{i-1}$
is \emph{fully assigned} on all $n_1$ edges.  This configuration is fed
through $K$ unmasked EIGN layers
(\cref{eq:eign_equ,eq:eign_inv})---no causal masking is needed because
every edge already has a definite spin value $\pm 1$.  A loop-level
representation is obtained by pooling the final equivariant edge features
over the edges of loop~$i$:
\begin{equation}
  \mathbf{h}_i = \frac{1}{|\bm l_i|}
  \sum_{e \in \bm l_i} X_{\text{equ},e}^{(K)},
  \label{eq:loop_pool}
\end{equation}
and the flip probability is
$p_i = \mathrm{sigmoid}\!\bigl(\mathrm{MLP}(\mathbf{h}_i)\bigr)$.

\paragraph{Directed-cycle gating.}
Before sampling $\alpha_i$, we check whether loop~$i$ is a
\textbf{directed cycle} in the current configuration~$\bm\sigma^{(i)}$
(all spins flow consistently around the loop).  If it is not, flipping
loop~$i$ would violate the ice rule (non-directed flips do not preserve
vertex charges), so the flip is gated off:
$\alpha_i \leftarrow 0$ deterministically.  This guarantees that every
sample satisfies the ice rule by construction.

\paragraph{Training.}
Because all ice states have $\calH = 0$, the variational free energy
reduces to pure entropy:
\begin{equation}
  F_\theta = T\cdot\mathbb{E}_{q_\theta}\!\bigl[\ln q_\theta(\bm\alpha)\bigr].
  \label{eq:modeA_loss}
\end{equation}
Minimizing $F_\theta$ drives $q_\theta$ toward the uniform distribution
over reachable ice states.  The gradient is estimated by REINFORCE with a
running-mean baseline~$b$:
\begin{equation}
  \nabla_\theta F_\theta
  \approx \frac{1}{M}\sum_{m=1}^{M}
  \bigl(\ln q_\theta(\bm\alpha_m) - b\bigr)\,
  \nabla_\theta \ln q_\theta(\bm\alpha_m),
  \label{eq:modeA_reinforce}
\end{equation}
where $\bm\alpha_m \sim q_\theta$ are autoregressive samples.  An entropy
bonus $-\eta\,H(q_\theta)$ is added to prevent premature mode collapse.

\textbf{Strengths:} every sample is a valid ice state by construction (zero
ice-rule violations); autoregressive sequence length is $\beta_1$, not
$n_1$; no causal masking is required.
\textbf{Limitation:} confined to a single charge sector; the basis-restricted
sequential sampling (\cref{sec:decision_tree}) reaches only
$|\varepsilon_{\mathbf{Q}}|$ states, which can be much fewer than
$|\calI_{\mathbf{Q}}|$ (the true number of ice states in the sector) due
to the directedness gating at each step
(\cref{sec:three_quantities,tab:counting}).  By \cref{thm:veblen_connectivity},
all $|\calI_{\mathbf{Q}}|$ states are connected by arbitrary directed cycle
moves, but the autoregressive basis-restricted sampler cannot exploit this.

\subsection{Mode B: EdgeMPVAN (direct-edge autoregressive sampling)}
\label{sec:modeB}

Mode~B directly samples all $n_1$ edge spins, permitting ice-rule
violations (monopole excitations) at finite temperature.  The joint
distribution is factored over a fixed edge ordering
$(e_1, e_2, \ldots, e_{n_1})$:
\begin{equation}
  q_\theta(\bm\sigma)
  = \prod_{k=1}^{n_1}
    q_\theta\!\bigl(\sigma_{e_k} \mid \sigma_{e_1},\ldots,\sigma_{e_{k-1}}\bigr).
  \label{eq:modeB_factor}
\end{equation}

\paragraph{Causal masking.}
To enforce the autoregressive property, edge~$e_k$ must receive messages
only from edges $e_j$ with $j < k$.  This is implemented by zeroing out
entries $(L)_{e_k,e_j}$ for $j \geq k$ in each EIGN operator, producing
lower-triangular versions of
\cref{eq:eign_equ,eq:eign_inv}.  At step~$k$, the
partially assigned spin vector $\tilde{\bm\sigma}$ has
$\tilde\sigma_{e_j} = \sigma_{e_j}$ for $j < k$ and
$\tilde\sigma_{e_j} = 0$ for $j \geq k$.  The masked EIGN layers process
this partial input, and the equivariant output at edge~$e_k$ is passed
through a sigmoid to produce
$p_k = \Pr(\sigma_{e_k} = +1 \mid \sigma_{<e_k})$.

\paragraph{Edge ordering.}
The ordering of edges affects conditioning quality.  BFS from a seed
vertex (spatially coherent: each new edge shares a vertex with an
already-assigned edge) is the recommended default; spectral ordering by
the Fiedler vector of $L_1^{\text{down}}$ and physics-informed strategies
(high-coordination edges first) are alternatives.

\paragraph{Training.}
The full variational free energy with a soft ice-rule penalty is
\begin{equation}
  F_\theta
  = \underbrace{\mathbb{E}_{q_\theta}\!\bigl[\calH(\bm\sigma)\bigr]}_{\text{energy}}
  + \underbrace{T\cdot\mathbb{E}_{q_\theta}\!
    \bigl[\ln q_\theta(\bm\sigma)\bigr]}_{\text{entropy}}
  + \underbrace{\lambda(T)\cdot\mathbb{E}_{q_\theta}\!
    \bigl[\|B_1\bm\sigma\|^2\bigr]}_{\text{ice-rule penalty}},
  \label{eq:modeB_loss}
\end{equation}
where $\calH = (\epsilon/2)\,\bm\sigma^T L_1^{\text{down}}\bm\sigma$ and
$\lambda(T) \propto 1/T$ strengthens the ice-rule constraint as
temperature decreases.  The energy and penalty terms both involve the same
operator $L_1^{\text{down}}$ and differ only in their prefactors
($\epsilon/2$ vs.\ $\lambda$).  Training uses a geometric temperature
schedule $T(t) = T_{\max}(T_{\min}/T_{\max})^{t/t_{\max}}$, starting from
a nearly uniform distribution at high~$T$ and annealing toward the
Boltzmann distribution at low~$T$.  Gradients are estimated by
REINFORCE as in Mode~A:
\begin{equation}
  \nabla_\theta F_\theta
  \approx \frac{1}{M}\sum_{m=1}^{M}
  \Bigl(\calH(\bm\sigma_m)
  + T\ln q_\theta(\bm\sigma_m)
  + \lambda\|B_1\bm\sigma_m\|^2 - b\Bigr)\,
  \nabla_\theta \ln q_\theta(\bm\sigma_m).
  \label{eq:modeB_reinforce}
\end{equation}

\textbf{Strengths:} full manifold access across all charge sectors;
finite-temperature thermodynamics with monopole excitations; the
equ$\to$inv channel ($|B_1|^T B_1$) serves as a real-time charge monitor,
tracking vertex charges during generation.
\textbf{Limitation:} ice-rule compliance is soft (not guaranteed); sampling
complexity scales with $n_1 \gg \beta_1$; causal masking of sparse EIGN
operators adds engineering complexity.

\subsection{Complementarity}

Mode~A provides exact $T = 0$ ice-manifold sampling within a charge sector.
Mode~B enables finite-$T$ Boltzmann sampling across the full phase space.
Together they cover the thermodynamic landscape: Mode~A for the ground-state
manifold (Pauling entropy, topological order), Mode~B for the thermal ensemble
(monopole statistics, phase transitions).

The structural parallel to MPVAN is worth noting: where MPVAN uses the
vertex Laplacian $L_0 = B_1 B_1^T$ to propagate information between vertex
spins coupled through edges, both modes use EIGN's edge Laplacian
$L_1^{\text{down}} = B_1^T B_1$ to propagate information between edge spins
coupled through shared vertices.  The transposition
$B_1 B_1^T \to B_1^T B_1$ reflects the shift from vertex-level to edge-level
spins; the Hamiltonian coupling structure is encoded in either case.  The
additional EIGN channels (inv$\to$inv, equ$\to$inv, inv$\to$equ) provide
geometric context---coordination numbers, unsigned charge
accumulation---that has no analogue in vertex-level MPVAN but may be
important for heterogeneous ASI lattices with mixed coordination
$z = 2, 3, 4$.

Whether any of this actually leads to samplers that improve on directed-loop
MCMC remains to be seen.  The TDL--ASI correspondence
(\cref{kr:charge,kr:hamiltonian,kr:hodge}) at least provides a clean
mathematical foundation for the attempt, and the counting table
(\cref{tab:counting}) gives a concrete measure of the effective state-space
complexity that any sampler must navigate.

% ===================================================================
% SECTION 9: SUMMARY
% ===================================================================
\section{Summary and Open Questions}
\label{sec:summary}

The first part of this document (Secs.~\ref{sec:notation}--\ref{sec:hamiltonian})
works out the mathematical dictionary between TDL cell complexes and the
ASI charge framework.  The six key results are:

\begin{enumerate}
  \item \textbf{Key Result~\ref{kr:S_matrix}} ($S$ matrix):
    the antisymmetric spin matrix is reconstructed from $B_1$ and
    $\bm\sigma$ via element-wise or skew-symmetrization formulas.
  \item \textbf{Key Result~\ref{kr:charge}} (Topological charge):
    $\mathbf{Q} = B_1\bm\sigma$; the ice rule is $\ker(B_1)$.
  \item \textbf{Key Result~\ref{kr:loop_basis}} (Loop-basis parameterization):
    ice states are parameterized by $\bm\alpha \in \{0,1\}^{\beta_1}$, with
    $|\calI_{\text{reachable}}| \leq 2^{\beta_1}$ per charge sector.
  \item \textbf{Key Result~\ref{kr:directed_cycle}} (Directed-cycle reduction):
    the actual reachable count $|\varepsilon_{\mathbf{q}}|$ can be orders of
    magnitude below $2^{\beta_1}$, varying by five orders of magnitude across
    the lattice zoo.
  \item \textbf{Key Result~\ref{kr:hodge}} (Hodge decomposition):
    $\ker(B_1) = \im(B_2) \oplus \ker(L_1)$, splitting the ice manifold into
    locally trivial and topologically protected modes.
  \item \textbf{Key Result~\ref{kr:hamiltonian}} (Hamiltonian):
    $\calH = (\epsilon/2)\,\bm\sigma^T L_1^{\text{down}}\bm\sigma$, connecting
    spin-ice energetics to the spectral theory of the Hodge Laplacian.
\end{enumerate}

The counting table (\cref{tab:counting}) is the most concrete result: it shows
that the directed-cycle constraint reduces the reachable ice-state count by
ratios ranging from $\sim$1:6 (Santa Fe $2\!\times\!2$ open) to
$\sim$1:170{,}000 (tetris $3\!\times\!3$ open).  This
reduction---invisible from $\beta_1$ alone---seems to be a structural
property of the lattice, though we do not yet have a good analytical
understanding of what controls it.

The second part (\cref{sec:outlook}) sketches the idea of combining VAN,
MPVAN, and EIGN into neural samplers for spin ice.  The key observation is
that EIGN's equivariant Laplacian $L_{\text{equ}} = B_1^T B_1$ is
identically the ASI Hamiltonian, so MPVAN's Hamiltonian-informed message
passing holds by construction.  Mode~A (loop-basis) and Mode~B (direct-edge)
are complementary approaches, but both are at a preliminary design stage.

\paragraph{Open questions and things we are unsure about.}
\begin{itemize}
  \item Whether the gap between $|\varepsilon_{\mathbf{q}}|$ and $|\calI|$
    on even-degree periodic lattices reflects a saturation ceiling of the
    multi-ordering enumeration method or a hard structural limit.
  \item Whether the directed-cycle gating in Mode~A limits the architecture
    too severely on lattices with very low
    $|\varepsilon_{\mathbf{q}}|/2^{\beta_1}$ ratios (e.g., tetris).
  \item How many EIGN layers are needed for the message-passing receptive
    field to span the largest loops in the cycle basis.
  \item Whether the causal masking required by Mode~B introduces artifacts
    or degrades the equivariance properties of the EIGN operators.
  \item Whether these neural samplers can actually outperform directed-loop
    MCMC on any practically relevant lattice size---we do not yet have
    evidence that they can.
  \item Both Mode~A and Mode~B use only $B_1$-based EIGN operators,
    passing messages between edges via shared vertices. The
    charge--current duality (\cref{sec:topological_current}) and the Hodge
    Hamiltonian (\cref{eq:hodge_hamiltonian}) suggest a natural
    extension: incorporating $B_2$-based operators that pass messages
    between edges via shared \emph{faces}, enabling the network to
    learn representations sensitive to plaquette current $J_f$ in
    addition to vertex charge $Q_v$. This would amount to lifting
    the architecture from a graph neural network to a simplicial or
    cell complex neural network operating on the full
    $(B_1, B_2)$ chain complex. We expect this direction to be
    relevant for systems where the $\kappa$ term is physically
    active (ring exchange, gauge theories), but implementing and
    validating Modes~A and~B on ASI systems with $B_1$-only
    operators is already a substantial undertaking and the more
    immediate goal.
\end{itemize}

We would welcome critical feedback on any of the formalism or the neural
sampling ideas, and especially on whether we have the ASI physics right.

% ===================================================================
% BIBLIOGRAPHY
% ===================================================================
\begin{thebibliography}{10}

\bibitem{morrison2013}
M.~J. Morrison, T.~R. Nelson, and C.~Nisoli,
``Unhappy vertices in artificial spin ice: new degeneracies from vertex
frustration,''
\emph{New J. Phys.} \textbf{15}, 045009 (2013).

\bibitem{nisoli2020}
C.~Nisoli,
``The concept of spin ice graphs and a field theory for their charges,''
\emph{AIP Advances} \textbf{10}, 115117 (2020).

\bibitem{caravelli2021}
F.~Caravelli, P.~Saccone, and C.~Nisoli,
``On the degeneracy of spin ice graphs, and its estimate via the Bethe
permanent,''
\emph{Proc. R. Soc. A} \textbf{477}, 20210108 (2021).

\bibitem{hajij2023topological}
M.~Hajij \emph{et al.},
``Topological deep learning: going beyond graph data,''
arXiv:2206.00606 (2023).

\bibitem{wu2019variational}
D.~Wu, L.~Wang, and P.~Zhang,
``Solving statistical mechanics using variational autoregressive networks,''
\emph{Phys. Rev. Lett.} \textbf{122}, 080602 (2019).

\bibitem{ma2024mpvan}
Q.~Ma, Z.~Ma, J.~Xu, H.~Zhang, and M.~Gao,
``Message passing variational autoregressive network for solving intractable
Ising models,''
\emph{Commun. Phys.} \textbf{7}, 236 (2024).

\bibitem{fuchsgruber2025}
D.~Fuchsgruber, T.~Postuvan, S.~G{\"u}nnemann, and S.~Geisler,
``Graph neural networks for edge signals: orientation equivariance and
invariance,''
In \emph{Proc. ICLR 2025}; arXiv:2410.16935.

\bibitem{papillon2024architectures}
M.~Papillon \emph{et al.},
``Architectures of topological deep learning: a survey of message-passing
topological neural networks,''
\emph{J. Mach. Learn. Res.} \textbf{25}, 1--75 (2024).

\bibitem{lieb1967}
E.~H. Lieb,
``Exact solution of the problem of the entropy of two-dimensional ice,''
\emph{Phys. Rev. Lett.} \textbf{18}, 692 (1967).

\bibitem{li2018deeper}
Q.~Li, Z.~Han, and X.-M. Wu,
``Deeper insights into graph convolutional networks for semi-supervised
learning,''
In \emph{Proc. AAAI 2018}, pp.\ 3538--3545.

\bibitem{wilson1974}
K.~G. Wilson,
``Confinement of quarks,''
\emph{Phys. Rev. D} \textbf{10}, 2445 (1974).

\bibitem{hermele2004}
M.~Hermele, M.~P.~A. Fisher, and L.~Balents,
``Pyrochlore photons: The $U(1)$ spin liquid in a $S=1/2$ three-dimensional
frustrated magnet,''
\emph{Phys. Rev. B} \textbf{69}, 064404 (2004).

\bibitem{castelnovo2008}
C.~Castelnovo, R.~Moessner, and S.~L. Sondhi,
``Magnetic monopoles in spin ice,''
\emph{Nature} \textbf{451}, 42 (2008).

\bibitem{veblen1912}
O.~Veblen,
``An application of modular equations in analysis situs,''
\emph{Ann. Math.} \textbf{14}, 86--94 (1912).

\bibitem{propp2002}
J.~Propp,
``Lattice structure for orientations of graphs,''
arXiv:math/0209005 (2002).

\bibitem{felsner2004}
S.~Felsner,
``Lattice structures from planar graphs,''
\emph{Electron. J. Combin.} \textbf{11}, R15 (2004).

\bibitem{whitney1932}
H.~Whitney,
``Non-separable and planar graphs,''
\emph{Trans. Amer. Math. Soc.} \textbf{34}, 339--362 (1932).

\end{thebibliography}

% ===================================================================
% APPENDICES
% ===================================================================
\appendix

\section{Worked Example: Triangle Graph \texorpdfstring{$K_3$}{K3}}
\label{app:triangle}

This appendix works through the $S$ matrix (\cref{kr:S_matrix}), topological
charge (\cref{kr:charge}), and Hamiltonian (\cref{kr:hamiltonian})
correspondences on the smallest nontrivial graph.

\subsection{Setup}

The triangle graph $K_3$ has $n_0 = 3$ vertices $\{v_0, v_1, v_2\}$ and
$n_1 = 3$ edges with reference orientations:
$e_0\!: v_0 \to v_1$,\;
$e_1\!: v_0 \to v_2$,\;
$e_2\!: v_2 \to v_1$.

\begin{figure}[h]
\centering
% Pre-compiled from tikz_triangle.tex (standalone)
\includegraphics[width=0.45\textwidth]{tikz_triangle}
\caption{Triangle $K_3$ with reference orientations (black arrows) and spin
configuration $\bm\sigma = (+1,+1,-1)^T$ (red dashed arrows showing physical
spin direction).  Vertex $v_0$ is a source ($Q=-2$), $v_1$ satisfies the
ice rule ($Q=0$), and $v_2$ is a sink ($Q=+2$).}
\label{fig:triangle}
\end{figure}

The incidence matrix is
\begin{equation}
  B_1 = \begin{pmatrix}
    -1 & -1 &  0 \\
    +1 &  0 & +1 \\
     0 & +1 & -1
  \end{pmatrix}
  \quad
  \begin{array}{l}
    \leftarrow v_0 \\ \leftarrow v_1 \\ \leftarrow v_2
  \end{array}
  \label{eq:B1_triangle}
\end{equation}
with columns indexed by $e_0, e_1, e_2$.  Each column has one $+1$ (head)
and one $-1$ (tail), and every column sums to zero.

Choose the spin configuration $\bm\sigma = (+1,\,+1,\,-1)^T$: spins on
$e_0$ and $e_1$ align with the reference orientation (pointing away from
$v_0$), while the spin on $e_2$ opposes it (pointing $v_1 \to v_2$).
This is a ``source'' configuration at $v_0$.

\subsection{\texorpdfstring{$S_{vv'}$}{S} matrix (Key Result~\ref{kr:S_matrix})}

Using $S_{ww'} = -[B_1]_{w,e}\,\sigma_e$ for adjacent $w, w'$ sharing
edge~$e$:
\begin{align*}
  S_{v_0 v_1} &= -[B_1]_{v_0,e_0}\,\sigma_{e_0}
  = -(-1)(+1) = +1, \\
  S_{v_0 v_2} &= -[B_1]_{v_0,e_1}\,\sigma_{e_1}
  = -(-1)(+1) = +1, \\
  S_{v_2 v_1} &= -[B_1]_{v_2,e_2}\,\sigma_{e_2}
  = -(-1)(-1) = -1.
\end{align*}
By antisymmetry, the full matrix is
\begin{equation}
  S = \begin{pmatrix}
     0 & +1 & +1 \\
    -1 &  0 & +1 \\
    -1 & -1 &  0
  \end{pmatrix}.
\end{equation}

\paragraph{Matrix-formula verification.}
We verify~\eqref{eq:Smatrix} by computing
$M = |B_1|\,D_\sigma\,B_1^T$ where $D_\sigma = \diag(+1,+1,-1)$:
\begin{equation}
  |B_1| = \begin{pmatrix}
    1 & 1 & 0 \\ 1 & 0 & 1 \\ 0 & 1 & 1
  \end{pmatrix},
  \quad
  D_\sigma B_1^T = \begin{pmatrix}
    -1 & +1 &  0 \\
    -1 &  0 & +1 \\
     0 & -1 & +1
  \end{pmatrix}.
\end{equation}
Then $M = |B_1|\,(D_\sigma B_1^T)$:
\begin{equation}
  M = \begin{pmatrix}
    -2 & +1 & +1 \\
    -1 &  0 & +1 \\
    -1 & -1 & +2
  \end{pmatrix}.
\end{equation}
The diagonal gives the charges:
$\diag(M) = (-2,\,0,\,+2) = \mathbf{Q}$.
The skew-symmetrization $\tfrac{1}{2}(M - M^T) = S$.~$\checkmark$

\subsection{Topological charge (Key Result~\ref{kr:charge})}

\begin{equation}
  \mathbf{Q} = B_1\bm\sigma
  = \begin{pmatrix} -1-1+0 \\ +1+0-1 \\ 0+1+1 \end{pmatrix}
  = \begin{pmatrix} -2 \\ 0 \\ +2 \end{pmatrix}.
\end{equation}
$v_0$ is a negative monopole (source, $Q = -2$); $v_1$ satisfies the ice
rule ($Q = 0$); $v_2$ is a positive monopole (sink, $Q = +2$).  Total
charge: $\sum_v Q_v = 0$.~$\checkmark$

\paragraph{Ice manifold.}
$\dim(\ker B_1) = n_1 - n_0 + 1 = 1$.  A basis vector is
$\bm\sigma_{\text{ice}} = (+1,\,-1,\,-1)^T$ (a clockwise circulation:
$v_0 \to v_1 \to v_2 \to v_0$, every vertex has one spin in and one out).
The full ice manifold on $K_3$ is
$\calI = \{(+1,-1,-1),\;(-1,+1,+1)\}$---the two circulations.

\paragraph{Coulomb classes.}
$K_3$ has $2^3 = 8$ spin configurations partitioned into four Coulomb
classes:
$C_{\mathbf{0}} = \{(\pm 1,\mp 1,\mp 1)\}$ (two ice states),
and three monopole classes of two configurations each.

\subsection{Hamiltonian (Key Result~\ref{kr:hamiltonian})}

\begin{equation}
  L_1^{\text{down}} = B_1^T B_1
  = \begin{pmatrix} 2 & 1 & 1 \\ 1 & 2 & -1 \\ 1 & -1 & 2 \end{pmatrix}.
\end{equation}

\textbf{Non-ice configuration} $\bm\sigma = (+1,+1,-1)^T$
($\mathbf{Q} = (-2,0,+2)^T$):
\begin{equation}
  \bm\sigma^T L_1^{\text{down}}\bm\sigma
  = \|\mathbf{Q}\|^2 = 4 + 0 + 4 = 8,
  \qquad
  \calH = \tfrac{\epsilon}{2}\cdot 8 = 4\epsilon.
\end{equation}

\textbf{Ice configuration} $\bm\sigma_{\text{ice}} = (+1,-1,-1)^T$:
\begin{equation}
  B_1\bm\sigma_{\text{ice}} = \mathbf{0},
  \qquad
  \calH_{\text{ice}} = 0.\;\checkmark
\end{equation}

\textbf{Spectrum of $L_1^{\text{down}}$.}  The eigenvalues are
$\{0, 3, 3\}$: one zero eigenvalue ($\beta_1 = 1$, the ice manifold) and
a doubly-degenerate spectral gap $\Delta = 3$.  The minimum monopole pair
energy is $\calH_{\min} = (\epsilon/2)\cdot\Delta\cdot c_{\min}^2$ where
$c_{\min}^2$ is the smallest nonzero squared expansion coefficient.


% -------------------------------------------------------------------
\section{Worked Example: \texorpdfstring{$3\times 3$}{3x3} Periodic Square Lattice}
\label{app:torus}

This appendix works through the full computation on a $3\times 3$ square
lattice with periodic boundary conditions (a discrete torus), producing
every matrix and eigenvalue explicitly.

\subsection{Lattice setup}

\textbf{Vertices:} 9 vertices on a $3\times 3$ grid, labeled $v_0,\ldots,v_8$
in row-major order with coordinates $v_k$ at
$(k\bmod 3,\;\lfloor k/3\rfloor)$.

\textbf{Periodic BCs:} Right wraps to left
($v_2 \sim v_0$, $v_5 \sim v_3$, $v_8 \sim v_6$);
top wraps to bottom
($v_6 \sim v_0$, $v_7 \sim v_1$, $v_8 \sim v_2$).
Topologically a torus~$T^2$.

\textbf{Edges:} $n_1 = 18$ (9~horizontal $+$ 9~vertical).
Reference orientation: horizontal edges point right, vertical edges point up.
Wrap-around edges: $e_2 (v_2\!\to\!v_0)$, $e_5 (v_5\!\to\!v_3)$,
$e_8 (v_8\!\to\!v_6)$, $e_{15} (v_6\!\to\!v_0)$,
$e_{16} (v_7\!\to\!v_1)$, $e_{17} (v_8\!\to\!v_2)$.

\textbf{Faces:} $n_2 = 9$ unit squares, all oriented counterclockwise.

\textbf{Counts:} $n_0 = 9$, $n_1 = 18$, $n_2 = 9$.  All vertices have
coordination $z_v = 4$.

\begin{figure}[h]
\centering
% Pre-compiled from tikz_torus.tex (standalone)
\includegraphics[width=0.65\textwidth]{tikz_torus}
\caption{The $3\times 3$ periodic square lattice (torus $T^2$).  Solid arrows
show the 12~interior edges with reference orientations (horizontal: right,
vertical: up).  Dotted gray arrows show the 6~wrap-around edges implementing
periodic boundary conditions.  Vertices are labeled $v_0$--$v_8$ in row-major
order.}
\label{fig:torus}
\end{figure}

\subsection{Incidence matrix \texorpdfstring{$B_1$}{B1} \texorpdfstring{$(9\times 18)$}{(9x18)}}

Using the convention $[B_1]_{v,e} = +1$ at the head, $-1$ at the tail:

{\small
\begin{equation}
B_1 = \left(\begin{array}{@{}rrrrrrrrrrrrrrrrrr@{}}
-1 &  0 &  1 &  0 &  0 &  0 &  0 &  0 &  0 & -1 &  0 &  0 &  0 &  0 &  0 &  1 &  0 &  0 \\
 1 & -1 &  0 &  0 &  0 &  0 &  0 &  0 &  0 &  0 & -1 &  0 &  0 &  0 &  0 &  0 &  1 &  0 \\
 0 &  1 & -1 &  0 &  0 &  0 &  0 &  0 &  0 &  0 &  0 & -1 &  0 &  0 &  0 &  0 &  0 &  1 \\
 0 &  0 &  0 & -1 &  0 &  1 &  0 &  0 &  0 &  1 &  0 &  0 & -1 &  0 &  0 &  0 &  0 &  0 \\
 0 &  0 &  0 &  1 & -1 &  0 &  0 &  0 &  0 &  0 &  1 &  0 &  0 & -1 &  0 &  0 &  0 &  0 \\
 0 &  0 &  0 &  0 &  1 & -1 &  0 &  0 &  0 &  0 &  0 &  1 &  0 &  0 & -1 &  0 &  0 &  0 \\
 0 &  0 &  0 &  0 &  0 &  0 & -1 &  0 &  1 &  0 &  0 &  0 &  1 &  0 &  0 & -1 &  0 &  0 \\
 0 &  0 &  0 &  0 &  0 &  0 &  1 & -1 &  0 &  0 &  0 &  0 &  0 &  1 &  0 &  0 & -1 &  0 \\
 0 &  0 &  0 &  0 &  0 &  0 &  0 &  1 & -1 &  0 &  0 &  0 &  0 &  0 &  1 &  0 &  0 & -1
\end{array}\right)
\label{eq:B1_torus}
\end{equation}
}

Each column has one $+1$ and one $-1$; each row has four nonzero entries
(two $+1$, two $-1$) reflecting $z_v = 4$.

\subsection{Incidence matrix \texorpdfstring{$B_2$}{B2} \texorpdfstring{$(18\times 9)$}{(18x9)}}

For face~$f_j$ with CCW boundary, $[B_2]_{e,f} = +1$ if edge~$e$ is
traversed in its reference direction, $-1$ if against.  For example,
$f_0$ has boundary $+e_0,+e_{10},-e_3,-e_9$:

{\small
\begin{equation}
B_2 = \left(\begin{array}{@{}rrrrrrrrr@{}}
 1 &  0 &  0 &  0 &  0 &  0 & -1 &  0 &  0 \\
 0 &  1 &  0 &  0 &  0 &  0 &  0 & -1 &  0 \\
 0 &  0 &  1 &  0 &  0 &  0 &  0 &  0 & -1 \\
-1 &  0 &  0 &  1 &  0 &  0 &  0 &  0 &  0 \\
 0 & -1 &  0 &  0 &  1 &  0 &  0 &  0 &  0 \\
 0 &  0 & -1 &  0 &  0 &  1 &  0 &  0 &  0 \\
 0 &  0 &  0 & -1 &  0 &  0 &  1 &  0 &  0 \\
 0 &  0 &  0 &  0 & -1 &  0 &  0 &  1 &  0 \\
 0 &  0 &  0 &  0 &  0 & -1 &  0 &  0 &  1 \\
-1 &  0 &  1 &  0 &  0 &  0 &  0 &  0 &  0 \\
 1 & -1 &  0 &  0 &  0 &  0 &  0 &  0 &  0 \\
 0 &  1 & -1 &  0 &  0 &  0 &  0 &  0 &  0 \\
 0 &  0 &  0 & -1 &  0 &  1 &  0 &  0 &  0 \\
 0 &  0 &  0 &  1 & -1 &  0 &  0 &  0 &  0 \\
 0 &  0 &  0 &  0 &  1 & -1 &  0 &  0 &  0 \\
 0 &  0 &  0 &  0 &  0 &  0 & -1 &  0 &  1 \\
 0 &  0 &  0 &  0 &  0 &  0 &  1 & -1 &  0 \\
 0 &  0 &  0 &  0 &  0 &  0 &  0 &  1 & -1
\end{array}\right)
\label{eq:B2_torus}
\end{equation}
}

Each column (face) has four nonzero entries; each row (edge) appears in
exactly two faces.

\subsection{Verification: \texorpdfstring{$B_1 B_2 = 0$}{B1 B2 = 0}}

We verify on face~$f_0$ (first column of $B_2$, nonzero at
$e_0 (+1)$, $e_3 (-1)$, $e_9 (-1)$, $e_{10} (+1)$).
For each vertex:
\begin{align*}
  v_0 &: (-1)(+1) + 0(-1) + (-1)(-1) + 0(+1) = -1+1 = 0, \\
  v_1 &: (+1)(+1) + 0(-1) + 0(-1) + (-1)(+1) = 1-1 = 0, \\
  v_3 &: 0(+1) + (-1)(-1) + (+1)(-1) + 0(+1) = 1-1 = 0, \\
  v_4 &: 0(+1) + (+1)(-1) + 0(-1) + (+1)(+1) = -1+1 = 0.
\end{align*}
All other vertices have no incident edges in~$f_0$.
The same holds for all columns: $B_1 B_2 = 0$.~$\checkmark$

\subsection{Graph Laplacian \texorpdfstring{$L_0$}{L0} and its spectrum}

With $z_v = 4$ everywhere, $L_0 = 4I_9 - A$.  The eigenvalues of the
$3\times 3$ torus are
\begin{equation}
  \lambda_{jk} = 2\bigl(2 - \cos(2\pi j/3) - \cos(2\pi k/3)\bigr),
  \quad j,k \in \{0,1,2\},
\end{equation}
giving spectrum $\{0,\, 3^{(\times 4)},\, 6^{(\times 4)}\}$.
Spectral gap $\lambda_1 = 3$; one zero eigenvalue confirms
$\beta_0 = 1$.~$\checkmark$

\subsection{Betti numbers and Euler characteristic}

Expected for the torus $T^2$: $\beta_0 = 1$, $\beta_1 = 2$, $\beta_2 = 1$.

\textbf{Euler characteristic:}
\begin{equation}
  \chi = n_0 - n_1 + n_2 = 9 - 18 + 9 = 0
  = \beta_0 - \beta_1 + \beta_2 = 1 - 2 + 1.\;\checkmark
\end{equation}

\textbf{Rank check:}
\begin{equation}
  \beta_1 = n_1 - \rank(B_1) - \rank(B_2) = 18 - 8 - 8 = 2.\;\checkmark
\end{equation}

\textbf{Ice manifold dimension:}
$\dim(\ker B_1) = 18 - 8 = 10$, which splits as
$\rank(B_2) + \beta_1 = 8 + 2 = 10$.~$\checkmark$

\subsection{Spectrum of \texorpdfstring{$L_1$}{L1}}

The eigenvalues of $L_1^{\text{down}} = B_1^T B_1$ on $\R^{18}$:
\begin{equation}
  \{0^{(\times 10)},\; 3^{(\times 4)},\; 6^{(\times 4)}\}.
\end{equation}
The 10 zero eigenvalues span $\ker(B_1)$ (the ice manifold).  The nonzero
eigenvalues match those of $L_0$ (standard property: $B_1^T B_1$ and
$B_1 B_1^T$ share nonzero spectra).

The eigenvalues of $L_1^{\text{up}} = B_2 B_2^T$ on $\R^{18}$:
\begin{equation}
  \{0^{(\times 10)},\; 3^{(\times 4)},\; 6^{(\times 4)}\}.
\end{equation}

The combined Hodge 1-Laplacian $L_1 = L_1^{\text{down}} + L_1^{\text{up}}$:
\begin{equation}
  \operatorname{spec}(L_1)
  = \{\underbrace{0,\, 0}_{\beta_1 = 2},\;
      \underbrace{3,3,3,3,3,3,3,3}_{8},\;
      \underbrace{6,6,6,6,6,6,6,6}_{8}\}.
\end{equation}
Spectral gap $\Delta_1 = 3$.

\subsection{Explicit Laplacians: all faces vs.\ no faces}

We now display the three Laplacians of~\eqref{eq:L0_def}--\eqref{eq:L2_def}
for the $3\times 3$ torus, comparing the two face-filling strategies.

\medskip\noindent
\textbf{Graph Laplacian $L_0 = B_1 B_1^T$\; $(9\times 9)$.}
Since every vertex has $z_v = 4$, $L_0 = 4I_9 - A$ where $A$ is the
adjacency matrix.  This matrix is the same under both face-filling
strategies (it depends only on the graph, not on faces):
{\small
\begin{equation}
L_0 = \left(\begin{array}{@{}rrrrrrrrr@{}}
 4 & -1 & -1 & -1 &  0 &  0 & -1 &  0 &  0 \\
-1 &  4 & -1 &  0 & -1 &  0 &  0 & -1 &  0 \\
-1 & -1 &  4 &  0 &  0 & -1 &  0 &  0 & -1 \\
-1 &  0 &  0 &  4 & -1 & -1 & -1 &  0 &  0 \\
 0 & -1 &  0 & -1 &  4 & -1 &  0 & -1 &  0 \\
 0 &  0 & -1 & -1 & -1 &  4 &  0 &  0 & -1 \\
-1 &  0 &  0 & -1 &  0 &  0 &  4 & -1 & -1 \\
 0 & -1 &  0 &  0 & -1 &  0 & -1 &  4 & -1 \\
 0 &  0 & -1 &  0 &  0 & -1 & -1 & -1 &  4
\end{array}\right)
\label{eq:L0_torus}
\end{equation}
}
The circulant block structure reflects the toroidal periodicity.
$\operatorname{spec}(L_0) = \{0,\, 3^{(\times 4)},\, 6^{(\times 4)}\}$;
one zero eigenvalue confirms $\beta_0 = 1$.

\medskip\noindent
\textbf{Face Laplacian $L_2 = B_2^T B_2$\; $(9\times 9)$, all faces filled.}
A direct computation yields
\begin{equation}
  L_2 = L_0 \quad\text{(identically).}
  \label{eq:L2_torus}
\end{equation}
This is \textbf{Poincar\'e duality} on the torus: the dual graph of the
$3\times 3$ periodic square lattice is again a $3\times 3$ periodic square
lattice, so the face adjacency structure is isomorphic to the vertex
adjacency structure.
$\operatorname{spec}(L_2) = \{0,\, 3^{(\times 4)},\, 6^{(\times 4)}\}$;
one zero eigenvalue confirms $\beta_2 = 1$.

With no faces filled, $L_2$ does not exist ($n_2 = 0$), and
$\beta_2 = 0$.

\medskip\noindent
\textbf{Hodge 1-Laplacian $L_1 = L_1^{\text{down}} + L_1^{\text{up}}$\;
$(18\times 18)$, all faces filled.}
The diagonal is uniformly~4 (each edge participates in~2 faces and
connects~2 vertices, each of degree~4, giving
$[L_1^{\text{down}}]_{ee} + [L_1^{\text{up}}]_{ee} = 2 + 2 = 4$).
Off-diagonal entries lie in $\{-1, 0, +1\}$:
{\scriptsize
\begin{equation}
L_1 = \left(\begin{array}{@{}rrrrrrrrrrrrrrrrrr@{}}
 4 &  0 & -1 &  0 & -1 &  0 & -1 &  0 &  0 &  0 &  0 &  0 &  1 &  0 &  0 &  0 &  0 &  0 \\
 0 &  4 &  0 & -1 &  0 &  1 &  0 & -1 &  0 &  0 &  0 &  0 &  0 & -1 &  0 &  0 &  0 &  0 \\
-1 &  0 &  4 &  0 & -1 &  0 &  0 &  0 & -1 &  0 &  0 &  0 &  0 &  0 &  1 &  0 &  0 &  0 \\
 0 & -1 &  0 &  4 &  0 &  1 &  0 &  0 &  0 & -1 &  0 &  0 &  0 &  0 &  0 & -1 &  0 &  0 \\
-1 &  0 & -1 &  0 &  4 &  0 &  0 &  0 &  0 &  0 & -1 &  0 &  0 &  0 &  0 &  0 &  1 &  0 \\
 0 &  1 &  0 &  1 &  0 &  4 &  0 &  0 &  0 &  0 &  0 & -1 &  0 &  0 &  0 &  0 &  0 & -1 \\
-1 &  0 &  0 &  0 &  0 &  0 &  4 &  0 & -1 &  0 & -1 &  0 &  1 &  0 &  0 &  0 &  0 &  0 \\
 0 & -1 &  0 &  0 &  0 &  0 &  0 &  4 &  0 & -1 &  0 &  1 &  0 & -1 &  0 &  0 &  0 &  0 \\
 0 &  0 & -1 &  0 &  0 &  0 & -1 &  0 &  4 &  0 & -1 &  0 &  0 &  0 &  1 &  0 &  0 &  0 \\
 0 &  0 &  0 & -1 &  0 &  0 &  0 & -1 &  0 &  4 &  0 &  1 &  0 &  0 &  0 & -1 &  0 &  0 \\
 0 &  0 &  0 &  0 & -1 &  0 & -1 &  0 & -1 &  0 &  4 &  0 &  0 &  0 &  0 &  0 &  1 &  0 \\
 0 &  0 &  0 &  0 &  0 & -1 &  0 &  1 &  0 &  1 &  0 &  4 &  0 &  0 &  0 &  0 &  0 & -1 \\
 1 &  0 &  0 &  0 &  0 &  0 &  1 &  0 &  0 &  0 &  0 &  0 &  4 &  0 & -1 &  0 & -1 &  0 \\
 0 & -1 &  0 &  0 &  0 &  0 &  0 & -1 &  0 &  0 &  0 &  0 &  0 &  4 &  0 & -1 &  0 &  1 \\
 0 &  0 &  1 &  0 &  0 &  0 &  0 &  0 &  1 &  0 &  0 &  0 & -1 &  0 &  4 &  0 & -1 &  0 \\
 0 &  0 &  0 & -1 &  0 &  0 &  0 &  0 &  0 & -1 &  0 &  0 &  0 & -1 &  0 &  4 &  0 &  1 \\
 0 &  0 &  0 &  0 &  1 &  0 &  0 &  0 &  0 &  0 &  1 &  0 & -1 &  0 & -1 &  0 &  4 &  0 \\
 0 &  0 &  0 &  0 &  0 & -1 &  0 &  0 &  0 &  0 &  0 & -1 &  0 &  1 &  0 &  1 &  0 &  4
\end{array}\right)
\label{eq:L1_torus}
\end{equation}
}

\noindent
The $6\times 6$ block structure (three $6\times 6$ blocks along the diagonal,
corresponding to the three rows of the lattice) reflects the toroidal
translation symmetry.

\medskip\noindent
\textbf{Face-filling comparison.}
The following table summarizes how the Laplacian spectra change between
the two face-filling strategies:
\begin{center}
\small
\begin{tabular}{lcc}
\toprule
& \textbf{All faces} ($n_2 = 9$) & \textbf{No faces} ($n_2 = 0$) \\
\midrule
$L_0 = B_1 B_1^T$ & \multicolumn{2}{c}{$\{0,\, 3^{(\times 4)},\, 6^{(\times 4)}\}$ \quad(unchanged)} \\
$L_1^{\text{down}} = B_1^T B_1$ & \multicolumn{2}{c}{$\{0^{(\times 10)},\, 3^{(\times 4)},\, 6^{(\times 4)}\}$ \quad(unchanged)} \\
$L_1^{\text{up}} = B_2 B_2^T$ & $\{0^{(\times 10)},\, 3^{(\times 4)},\, 6^{(\times 4)}\}$ & $\mathbf{0}$ \\
$L_1 = L_1^{\text{down}} + L_1^{\text{up}}$ & $\{0^{(\times 2)},\, 3^{(\times 8)},\, 6^{(\times 8)}\}$ & $\{0^{(\times 10)},\, 3^{(\times 4)},\, 6^{(\times 4)}\}$ \\
$L_2 = B_2^T B_2$ & $\{0,\, 3^{(\times 4)},\, 6^{(\times 4)}\}$ & does not exist \\
\midrule
$\beta_1 = \dim\ker L_1$ & \textbf{2} & \textbf{10} \\
\bottomrule
\end{tabular}
\end{center}

\noindent
The critical observation: $L_1^{\text{down}}$ and $L_1^{\text{up}}$ each
have~10 zero eigenvalues individually, but their sum $L_1$ has only~2.
Adding $L_1^{\text{up}}$ ``lifts'' 8~of the~10 zero modes of
$L_1^{\text{down}}$, because these modes lie in $\im(B_2)$ (they are
boundaries of face signals) and $L_1^{\text{up}}$ acts nontrivially on them.
The 2~surviving zero modes are exactly the harmonic modes---they lie in
$\ker(B_1) \cap \ker(B_2^T)$, the intersection invisible to both Laplacians.
With no faces, $L_1^{\text{up}} = 0$ and all~10 cycles remain
in~$\ker(L_1)$.

\subsection{The two harmonic modes}

The two harmonic eigenvectors $\mathbf{h}_H, \mathbf{h}_V \in \ker(L_1)$
correspond to the two non-contractible cycles on the torus:

\textbf{Horizontal winding mode:}
$[\mathbf{h}_H]_e = 1/3$ for horizontal edges, $0$ for vertical.
(Normalized: $\|\mathbf{h}_H\|^2 = 9\cdot(1/3)^2 = 1$.)

\textbf{Vertical winding mode:}
$[\mathbf{h}_V]_e = 0$ for horizontal edges, $1/3$ for vertical.

Verification: $B_1\mathbf{h}_H = \mathbf{0}$ (at each vertex, one
horizontal edge enters and one leaves due to periodicity).
$B_2^T\mathbf{h}_H = \mathbf{0}$ (each face has one horizontal edge
traversed forward and one backward, canceling).  Therefore
$L_1\mathbf{h}_H = \mathbf{0}$.~$\checkmark$

These two modes are orthogonal and span $\ker(L_1)$.  They represent
topologically protected degrees of freedom: no sequence of local plaquette
flips can change the winding numbers.

\subsection{Ice configuration and Hodge decomposition}

The ``all-right, all-up'' configuration has
$\bm\sigma_0 = (1,1,1,1,1,1,1,1,1,\;1,1,1,1,1,1,1,1,1)^T$.

\textbf{Ice rule:} At each vertex, two edges point in and two point out,
so $B_1\bm\sigma_0 = \mathbf{0}$.~$\checkmark$

\textbf{Hodge projection:}
$\langle\bm\sigma_0, \mathbf{h}_H\rangle = 9\cdot(1/3) = 3$,\;
$\langle\bm\sigma_0, \mathbf{h}_V\rangle = 3$.  So
$\bm\sigma_{\text{harm}} = 3\mathbf{h}_H + 3\mathbf{h}_V = \bm\sigma_0$.
The curl component is $\bm\sigma_{\text{curl}} = \mathbf{0}$: this
configuration is \emph{purely harmonic}.  It has nonzero winding number in
both directions and no local curl structure.

\subsection{A monopole excitation}

Starting from $\bm\sigma_0$, flip $e_0$ ($v_0 \to v_1$):
set $\sigma_{e_0} = -1$.  Call this $\bm\sigma_1$.
\begin{equation}
  \mathbf{Q} = B_1\bm\sigma_1
  = B_1\bm\sigma_0 - 2\,B_1\mathbf{e}_0
  = -2\begin{pmatrix} -1\\1\\0\\\cdots\\0 \end{pmatrix}
  = \begin{pmatrix} +2\\-2\\0\\\cdots\\0 \end{pmatrix}.
\end{equation}
A monopole--antimonopole pair: $Q_{v_0} = +2$, $Q_{v_1} = -2$, all others
zero.

\textbf{Energy:}
$\calH[\bm\sigma_1] = (\epsilon/2)\|\mathbf{Q}\|^2 = (\epsilon/2)(4+4)
= 4\epsilon$.  The ice configuration had $\calH = 0$, so the monopole
pair costs~$4\epsilon$.

\subsection{Pauling entropy}

The Pauling approximation for uniform $z = 4$:
\begin{equation}
  |\calI_{\text{P}}|
  = 2^{18}\!\left(\frac{\binom{4}{2}}{2^4}\right)^{\!9}
  = 2^{18}\!\left(\frac{3}{8}\right)^{\!9}
  \approx 38.4.
\end{equation}
Pauling entropy per vertex:
$s_{\text{P}} = \frac{1}{9}\ln 38.4 \approx 0.405$,
close to Lieb's exact result~\cite{lieb1967}
$s_{\text{Lieb}} = \tfrac{3}{2}\ln(4/3) \approx 0.431$.
The small discrepancy is a finite-size effect.

\subsection{Loop basis and ice-state counting}

With no faces filled ($n_2 = 0$):
$\beta_1 = \dim(\ker B_1) = 18 - 9 + 1 = 10$,
giving an upper bound of $2^{10} = 1024$ from the $\bm\alpha$
parameterization (\cref{thm:gf2_bijection}).  Brute-force enumeration of all
$2^{18}$ configurations confirms $|\calI| = 148$ ice states (all in a single
charge sector since all vertices have even coordination $z = 4$).  Thus
$|\calI| = |A_{\mathbf{Q}}| = 148$, verifying the GF(2) bijection.

The multi-ordering sequential enumeration ($K = 200$) finds
$|\varepsilon_{\mathbf{Q}}| = 101$ states, confirming the strict hierarchy
$101 < 148 < 1024$.  The 47 ``missing'' states require simultaneous
multi-cycle flips (\cref{sec:three_quantities}): their $\bm\alpha$ vectors
involve basis cycles that are individually non-directed from the seed,
but whose combined flip sets preserve charge at every vertex.

The Pauling estimate is $\approx 38$, so
$|\calI|/2^{\beta_1} = 148/1024 \approx 14.5\%$ reflects the fraction of
$\bm\alpha$ vectors that produce valid ice states.

With all 9 faces filled ($n_2 = 9$): the faces account for 8 of the 10
independent cycles (9~faces minus one linear relation), leaving
$\beta_1 = 10 - 8 = 2$.  The two surviving harmonic modes are exactly the
horizontal and vertical winding modes from the previous subsection---the
non-contractible cycles that wrap around the torus.  This illustrates the
two extremes of the face-filling strategy: with no faces, $\beta_1 = 10$
and the full cycle space is available for loop-flip sampling; with all
faces, $\beta_1 = 2$ and only the global winding modes survive.

\end{document}
